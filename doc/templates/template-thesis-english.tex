% Example file for the class ``univie-ling-thesis''
% Run pdflatex -- biber -- pdflatex
\documentclass[naustrian,english]{univie-ling-thesis}

\usepackage{babel}

% Use the PDF/A1-b standard as required by the Hochschulschriften-Service
% (see http://e-theses.univie.ac.at/elektronische_einreichung.html)
\usepackage[a-1b]{pdfx}
\hypersetup{pdfborder={0 0 0}}

% By default, the class loads the biblatex package
% with Unified style and biber backend.
% If you want to use classic bibtex (with unified.bst),
% use the class option ``biblatex=false''
\bibliography{biblatex-examples}

% Personal data
\author{First Name Surname, B.A. M.A.}
\studienkennzahl{A 792 327}
\studienrichtung{Sprachwissenschaft}

% Joint work:
%\author{First Name Surname, B.A. M.A. \and First Name Surname, B.A. M.A.}
%\studienkennzahl{A 066 899 \and A 066 899}

% Paper-related data
\thesistype{diss}% valid options: magister, diplom, bachelor, master, diss
\title{Title of the Thesis}
\subtitle{Subtitle}
\volume{1}{5}% Vol. x of y volumes
% The suitable degree is automatically set, but you can override
% it by uncommenting the following command:
% (note that female forms of degrees, where appropriate, are
%  used if you use the class option ``fdegree=true'')
%\degree{Doktor der Philosophie (Dr. phil.)}% Targeted degree
\supervisor{Univ.-Prof. Dr. First Name Surname}
\cosupervisor{Univ.-Prof. Dr. First Name Surname}


\begin{document}

% The title page
\maketitle

\tableofcontents

\chapter{General hints}\label{cha:hints}

It is recommended that you use \textsf{biblatex} for consistent references in Unified style
(see \foreignlanguage{naustrian}{\emph{Leitfaden für die Gestaltung von schriftlichen Arbeiten und Unterlagen}; März 2018}).
For normal inline references, use \verb|\textcite|: \textcite[22]{sarfraz}; for references in parentheses use \verb|\parencite|: \parencite{sarfraz}.

For quotation marks and quotations, use the macros provided by the package \emph{csquotes}: \enquote{double quotation marks}, \enquote*{single quotation marks},
\enquote{an \enquote{embedded} quotation}. Quotation with references can be produced with \verb|\textquote| and \verb|\textcquote|:
\textquote[{\cite[222]{sarfraz}}]{A quote with reference}, even simpler \textcquote[222]{sarfraz}{A quote with reference}.

For longer quotations, use \emph{displayquote} or \emph{displaycquote}:

\begin{displayquote}[{\cite[246]{sarfraz}}]
	A long long long long long long long long long long long long long long long long long long
	long long long long long long long long long long long long long long long long long long long long
	long long long long long long long long long long long long long long long long long long long long
	quotation.
\end{displayquote}

\begin{displaycquote}[246]{sarfraz}
	A long long long long long long long long long long long long long long long long long long
	long long long long long long long long long long long long long long long long long long long long
	long long long long long long long long long long long long long long long long long long long long
	quotation.
\end{displaycquote}
%
There are also macros for omissions in quotations: \textelp{}, insertions: \textins{my insertion} and combinations of both: \textelp{insertion after omission},
\textelp*{insertion before omission}.

Use the semantic markup provided by the class instead of manual shape change:
\begin{itemize}
	\item Expressions (object language) in italics: The word \Expression{Wort}.
	\item Meaning is given in single quotes: \Meaning{meaning}
	\item Semantic concepts in small caps: the concept \Concept{concept}
\end{itemize}
%
Numbered linguistic examples can be produced by means of the \emph{example} environment (for multi-line examples) and \emph{examples} environment
(for subsequent single-line examples):

\begin{example}
	This is a multi-line example
	
	It might consist of several paragraphs
\end{example}

\begin{examples}
	\item This is a single-line example\label{exa:example-single}
	\item Each paragraph gets its own number
\end{examples}
%
It is recommended to refer to the examples like this: \prettyref{exa:example-single}. Also to chapters, incidentally: \prettyref{cha:hints}.

% The bibliography
\printbibliography[heading=bibnumbered]

\end{document}
