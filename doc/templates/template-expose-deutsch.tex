% Beispieldatei für die Klasse ``univie-ling-expose''
% Übersetzen mit pdflatex -- biber -- pdflatex
\documentclass[naustrian]{univie-ling-expose}

\usepackage{babel}

% Standardmäßig wird das Biblatex-Paket mit Unified-Stil und
% Biber-Backend geladen.
% Wenn Sie klassisches Bibtex (mit unified.bst) verwenden wollen,
% verwenden Sie die Klassenoption ``biblatex=false''
\bibliography{biblatex-examples}

% Persönliche Angaben
\author{Vorname Nachname, B.A. M.A.}
\studienkennzahl{A 792 327}
\studienrichtung{Sprachwissenschaft}

% Arbeitsrelevante Angaben
\title{Titel der Arbeit}
\subtitle{Untertitel}
\supervisor{Univ.-Prof. Dr. Vorname Name}
\cosupervisor{Univ.-Prof. Dr. Vorname Name}
% Advisory board (Doktoratsbeirat)
\advisor{Univ.-Prof. Dr. Vorname Name}


\begin{document}

% Die Titelseite
\maketitle

% Inhaltsverzeichnis
\tableofcontents

\section{Einleitung}\label{sec:einleitung}

Worum geht es in Ihrem Forschungsprojekt? Erklären Sie das Thema und seine Relevanz für das Fach.
Skizzieren Sie kurz die Gliederung des Exposés.


\section{Forschungsstand}\label{sec:forschungsstand}

Diskutieren Sie die vorhandene Literatur, machen Sie Lücken aus und okkupieren Sie diese!


\section{Forschungsfragen}\label{sec:forschungsfragen}

Nachdem Sie die Nischen, die Sie okkupieren möchten, lokalisiert haben, formulieren Sie Ihre Forschungsfragen.
Seien Sie möglichst klar und präzise. 


\section{Daten und Methoden}\label{sec:daten}

Erläutern Sie, welche Daten Sie analysieren wollen, um die Fragen zu klären. Seien Sie auch hier so präzise wie möglich.
Begründen Sie, welche Methoden Sie wählen, um die Daten zu erschließen. Stellen Sie auch methodische und empirische
Beschränkungen klar heraus.


\section{Betreuer/innen}\label{sec:betreuer}

Erläutern Sie kurz, warum Sie Ihre Betreuer/innen gewählt haben. 


\section{Schluss}\label{sec:schluss}

Fassen Sie ihre Ideen kurz zusammen, kommen Sie nochmals auf die Forschungsfragen zurück.


\section{Zeitplan}\label{sec:zeitplan}

Skizzieren Sie Ihren vorläufigen Zeitplan in tabellarischer Form.


\section{Tipps}

Verwenden Sie am Besten \textsf{biblatex} für konsistente Literaturverweise im \emph{Unified}-Stil
(vgl. \emph{Leitfaden für die Gestaltung von schriftlichen Arbeiten und Unterlagen}; März 2018).
Für normale Verweise im Text \verb|\textcite|: \textcite[22]{brandt}, für Verweise in Klammern \verb|\parencite|: \parencite{brandt}.

Für Anführungszeichen und Zitate verwenden Sie am Besten die Befehle des Paketes \emph{csquotes}: \enquote{doppelte Anführungszeichen},
\enquote*{einfache Anführungszeichen}, \enquote{ein \enquote{Zitat} im Zitat}. Für Zitate mit Literaturverweis gibt es 
\verb|\textquote| bzw. \verb|\textcquote|: \textquote[{\cite[202]{spiegelberg}}]{Ein Zitat mit Verweis}, noch einfacher 
\textcquote[202]{spiegelberg}{Ein Zitat mit Verweis}.

Für längere Zitate verwenden Sie \emph{displayquote} oder \emph{displaycquote}:

\begin{displayquote}[{\cite[202]{spiegelberg}}]
	Ein langes langes langes langes langes langes langes langes langes langes langes langes langes langes langes langes langes langes
	langes langes langes langes langes langes langes langes langes langes langes langes langes langes langes langes langes langes langes langes
	langes langes langes langes langes langes langes langes langes langes langes langes langes langes langes langes langes langes langes langes
	Zitat.
\end{displayquote}

\begin{displaycquote}[202]{spiegelberg}
	Ein langes langes langes langes langes langes langes langes langes langes langes langes langes langes langes langes langes langes
	langes langes langes langes langes langes langes langes langes langes langes langes langes langes langes langes langes langes langes langes
	langes langes langes langes langes langes langes langes langes langes langes langes langes langes langes langes langes langes langes langes
	Zitat.
\end{displaycquote}
%
Es gibt auch Makros für Auslassungen: \textelp{} und für Einfügungen in Zitaten: \textins{meine Einfügung} bzw. für beides kombiniert: \textelp{Einfügung nach Auslassung}, \textelp*{Einfügung vor Auslassung}.

Verwenden Sie semantisches Markup statt manueller Textauszeichnung:
\begin{itemize}
	\item Ausdruck (Objektsprachliches) kursiv: Das Wort \Expression{Wort}.
	\item Bedeutungsangaben in einfachen Anführungszeichen: \Meaning{Bedeutung}
	\item Semantische Konzepte in Kapitälchen: Das Konzept \Concept{Konzept}
\end{itemize}
%
Nummerierte linguistische Beispiele bekommen Sie mit der Umgebung \emph{example} (für mehrzeilige Beispiele) und \emph{examples} (für einzeilige):

\begin{example}
	Das ist ein mehrzeiliges Beispiel
	
	Es kann mehrere Absätze enthalten
\end{example}

\begin{examples}
	\item Das ist ein einzeiliges Beispiel\label{exa:Beispiel-einzeilig}
	\item Jeder Absatz wird in diesem Stil neu nummeriert
\end{examples}
%
Auf die Beispiele verweist man am Besten so: \prettyref{exa:Beispiel-einzeilig}. Übrigens auch auf Abschnitte\footnote{Denken Sie dran: Exposés haben, wie Artikel, keine \emph{Kapitel}, sondern nur \emph{Abschnitte}. Kapitel haben Bücher!}: \prettyref{sec:daten}.

% Die Bibliographie
\printbibliography[heading=bibnumbered]

\end{document}
