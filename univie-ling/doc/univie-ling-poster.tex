% !TeX spellcheck = en_US
\documentclass[english]{article}

\usepackage[osf]{libertine}
\usepackage[scaled=0.7]{beramono}
\usepackage[T1]{fontenc}
\usepackage[latin9]{inputenc}
\usepackage[hyphens]{url}
\usepackage{csquotes}
\usepackage[pdfusetitle,
 bookmarks=true,bookmarksnumbered=false,bookmarksopen=false,
 breaklinks=false,pdfborder={0 0 0},backref=false,colorlinks=false]
 {hyperref}

% Tweak the TOC (make it more compact)
\usepackage{tocloft}
\setlength{\cftaftertoctitleskip}{6pt}
\setlength{\cftbeforesecskip}{0pt}
\setlength{\cftbeforesubsecskip}{0pt}
\renewcommand{\cfttoctitlefont}{\normalsize\bfseries}
\renewcommand{\cftsecfont}{\small\bfseries}
\renewcommand{\cftsecpagefont}{\small\bfseries}
\renewcommand{\cftsubsecfont}{\small}
\renewcommand{\cftsubsecpagefont}{\small}

% markup
\newcommand*\jmacro[1]{\textbf{\texttt{#1}}}
\newcommand*\jcsmacro[1]{\jmacro{\textbackslash{#1}}}
\newcommand*\joption[1]{\textbf{\texttt{#1}}}
\newcommand*\jfmacro[1]{\texttt{#1}}
\newcommand*\jfcsmacro[1]{\jfmacro{\textbackslash{#1}}}

% macros
\newcommand*\uvlt{\textsf{univie-ling-poster}}
\providecommand{\LyX}{\texorpdfstring{L\kern-.1667em\lower.25em\hbox{Y}\kern-.125emX\@}{LyX}}

% improve layout
\tolerance 1414
\hbadness 1414
\emergencystretch 1.5em
\hfuzz 0.3pt
\widowpenalty = 10000
\vfuzz \hfuzz
\raggedbottom

% Conditional pagebreak
\def\condbreak#1{%
	\vskip 0pt plus #1\pagebreak[3]\vskip 0pt plus -#1\relax}

\usepackage{microtype}

\usepackage{babel}

\usepackage{listings}
\lstset{language={[LaTeX]TeX},
        basicstyle={\small\ttfamily},
        frame=single}

\setcounter{tocdepth}{2}

\begin{document}

\title{The \uvlt\ class}

\author{\texorpdfstring{J�rgen Spitzm�ller%
\thanks{Please report issues via \protect\url{https://github.com/jspitz/univie-ling}.}}{J�rgen Spitzm�ller}}

\date{Version 2.2, 2022/12/06}

\maketitle

\begin{abstract}
\noindent The \uvlt\ class provides a \LaTeXe\ class suitable for scientific posters in class or at conferences.
The class adheres to the corporate design of the University of Vienna.%
\footnote{\raggedright\url{https://communications.univie.ac.at/fileadmin/user_upload/d_oeffentlichkeitsarbeit/Dokumente/UniversitaetWien_CD_Manual_Mai_2022_interaktiv.pdf}.}
Therefore, although this class has been written for students in the Department of Linguistics, it might also be useful for other fields
and for students and researchers alike.
This manual documents the class.
\end{abstract}

\tableofcontents

\section{Aims and scope}

The \uvlt\ class provides a template for scientific posters that follows the corporate design of University of Vienna, using
the logo of the university and employing the design recommendations for (non-scientific) posters (regarding fonts, colors,
and layout).

Posters are a meanwhile established genre for knowledge transfer. They are used, above all, to present research in a condensed
form (in so-called \enquote*{poster sessions}). As opposed to other academic genres, they should be \enquote*{catchy} and
quickly consumable. Hence, they are heavily graphic-based.

This class aims to help creating a \enquote*{catchy} but coherently designed poster which presents information in
graphically segmented, easily navigatable chunks. It supports posters from A0 to A4 format in landscape and portrait orientation,
eases the use of multiple columns and provides a range of colored boxes for the presentation of information chunks, thereby
again following the color scheme of University of Vienna corporate design.

The class builds on \textsf{beamerposter}, a \LaTeX\ package that uses the \textsf{beamer} class to generate posters \cite{beamerposter}.

\section{Requirements of \uvlt}\label{sec:req-jslp}

The following class and packages are required and loaded by \uvlt:
\begin{itemize}
 \setlength\itemsep{0pt}
 \item \textsf{beamer}: Beamer class (base class).
 \item \textsf{beamerposter}: Unsing beamer for scientific posters.
 \item \textsf{csquotes}: Context sensitive quotations.
 \item \textsf{graphicx}: Graphic support.
 \item \textsf{geometry}: Page layout settings.
 \item \textsf{l3keys}: Key-value interface for class options.
 \item \textsf{translator}: Localization machinery.
 \item \textsf{url}: Support for typesetting URLs.
\end{itemize}
The following packages are required for specific features and loaded by default. However, the loading can be individually and generally omitted (see sec.~\ref{coptions}):
\begin{itemize}
 \setlength\itemsep{0pt}
 \item \textsf{sourceserifpro}: Default serif font (\emph{Source Serif Pro}).
 \item \textsf{sourcesanspro}: Default sans serif font (\emph{Source Sans Pro}).
 \item \textsf{sourcecodepro}: Default monospaced font (\emph{Source Code Pro}).
 \item \textsf{tcolorbox}: Fancy colored boxes.
 \item \textsf{babel}: Multilingual support.
 \item \textsf{biblatex}: Contemporary bibliography support.
 \item \textsf{caption}: Caption layout adjustments.
 \item \textsf{covington}: Support for linguistic examples\slash glosses.
 \item \textsf{fontenc}: Set the font encoding for PostScript fonts. Loaded with option \joption{T1}.
 \item \textsf{microtype}: Micro-typographic adjustments.
 \item \textsf{prettyref}: Verbose cross-references.
 \item \textsf{tcolorbox}: Fancy colored boxes.
\end{itemize}\condbreak{3\baselineskip}
The following packages are required for optional features (not used by default):
\begin{itemize}
 \setlength\itemsep{0pt}
 \item \textsf{biblatex-apa}: APA style for \textsf{biblatex}.
 \item \textsf{draftwatermark}: Create a draft mark.
 \item \textsf{fontspec}: Load OpenType fonts (with LuaTeX or XeTeX).
 \item \textsf{polyglossia}: Multi-language and script support.
\end{itemize}

\section{Fonts}\label{fonts}

The class uses, by default, PostScript (a.\,k.\,a. Type\,1) fonts and thus requires classic (PDF)LaTeX. Optionally, however, you can also
use OpenType fonts via the \textsf{fontspec} package and the XeTeX or LuaTeX engine instead.
In order to do this, use the class option \joption{fonts=otf} (see sec.~\ref{coptions} for details).

In both cases, the class uses by default \emph{Source Serif Pro} as a serif font, \emph{Source Sans Pro} as a sans serif font, and \emph{Source Code Pro}
as a monospaced (typewriter) font. This font is included in most \LaTeX\ distributions by default.
If you use \joption{fonts=otf} with XeTeX, you just have to make sure that you have the fonts \emph{Source Serif Pro}, \emph{Source Sans Pro}
and \emph{Source Code Pro} installed on your operating system (with exactly these names!).

Note that by default, with PostScript fonts, \uvlt\ also loads the \textsf{fontenc} package with T1 font encoding, but this can be customized
(see sec.~\ref{coptions} for details).

If you want (or need) to load all fonts and font encodings manually, you can switch off all automatic loading of fonts and font encodings by
the class option \joption{fonts=none} (see sec.~\ref{coptions}).


\section{Class Options}\label{coptions}

The \uvlt\ class provides a range of key=value type options to control the font handling, package loading and some specific behavior.
These are documented in this section.

\subsection{Font selection}

As elaborated above, the class supports PostScript fonts (via LaTeX and PDFLaTeX) as well as OpenType fonts (via XeTeX and LuaTeX).
PostScript is the traditional LaTeX font format. Specific LaTeX packages and metrics files are needed to use the fonts (but all fonts
needed to use this class should be included in your LaTeX distribution and thus ready to use). OpenType fonts, by contrast, are
taken directly from the operating system. They usually provide a wider range of glyphs, which might be a crucial factor for a linguistic
handout. However, they can only be used by newer TeX engines such as XeTeX and LuaTeX.\condbreak{2\baselineskip}

The class provides the following option to set the font handling:
\begin{description}
 \setlength\itemsep{0pt}
 \item{\joption{fonts=ps|otf|none}}: if \joption{ps} is selected, PostScript fonts are used (this is the default and
       the correct choice if you use LaTeX or PDFLaTeX); if \joption{otf} is selected, OpenType fonts are used, the class
       loads the \textsf{fontspec} package, sets \emph{Source Serif Pro} as main font, \emph{Source Sans Pro} as sans serif font,
       and \emph{Source Code Pro} as monospaced font (this is the correct choice if you use XeTeX or LuaTeX; make sure you have
       the respective fonts installed on your system);
       if \joption{none} is selected, finally, the class will not load any font package at all, and neither \textsf{fontenc}
       (this choice is useful if you want to control the font handling completely yourself).
\end{description}
%
The base font size can be set via 
\begin{description}
	\setlength\itemsep{0pt}
	\item{\joption{fontsize=<size>}}
\end{description}
%
With PostScript fonts, \uvlt\ also loads the \textsf{fontenc} package with T1 font encoding, which is suitable for
most Western European (and some Eastern European) writing systems. In order to load different, or more, encodings, the class option
\begin{description}
	\setlength\itemsep{0pt}
	\item{\joption{fontenc=<encoding(s)>}} can be used (e.\,g., \joption{fontenc=\{T1,X2\}}).
	With \joption{fontenc=none}, the loading of the \textsf{fontenc} package can be prevented. The package is also not loaded with
	\joption{fonts=none}.
\end{description}


\subsection{Layout settings}

The following layout options are available
\begin{description}
	\setlength\itemsep{0pt}
	\item{\joption{cd=german|english}} Select either German or English corporate design (independent of the document language). 
	      This currently only affects the department name.
	\item{\joption{pagesize=a0|a1|a2|a3|a4}} Set the page size. Default is \texttt{a0}.
	\item{\joption{portrait=true|false}} Set the handout to portrait format (default is landscape).
	      This also adjusts the logo positioning.
	\item{\joption{scale=<double>}} Scales the contents of the poster. The default is \texttt{1.4}.
\end{description}

\subsection{Language setting}

By default, the \uvlt\ loads the \textsf{babel} package. In order to set the language(s) of your poster, you should pass them
as class options, using the main language last (e.\,g., \texttt{german, english}). The class provides some predefined strings
for German and English (if you want to add other languages, feel free to contact me).

If you need \textsf{polyglossia} rather than \textsf{babel} for language support, please do not use the package yourself, but
rather use the package option \joption{polyglossia=true}. This assures correct loading order. This also presets \joption{fonts=otf}.
With \textsf{polyglossia}, languages are setup by means of dedicated commands (\verb|\setmainlanguage|, \verb|\setotherlanguage|).

\subsection{Package loading}\label{packageloading}

Most of the extra features provided by the class can be switched off. This might be useful if you do not need the respective feature anyway,
and crucial if you need an alternative package that conflicts with one of the preloaded package.

All following options are \joption{true} by default. They can be switched off one-by-one via the value \joption{false}, or altogether, by means of the special option \joption{all=false}. You can also switch selected packages on\slash off again after this
option (e.\,g., \joption{all=false,microtype=true} will switch off all packages except \textsf{microtype}).

\begin{description}
 \setlength\itemsep{0pt}
 \item{\joption{apa=true|false}}: If \joption{true}, the \textsf{biblatex-apa} style is used when \textsf{biblatex} is loaded. 
        By default, the included \textsf{univie-ling} style is loaded, instead. See sec.~\ref{bibliography} for details.
 \item{\joption{biblatex=true|false}}: If \joption{false}, \textsf{biblatex} is not loaded. This is useful if you prefer Bib\TeX\
        over \textsf{biblatex}, but also if you neither want to use the preloaded \textsf{univie-ling} style nor the
        alternative \textsf{biblatex-apa} style (i.\,e., if you want to load \textsf{biblatex}  manually with different options).
        See sec.~\ref{bibliography} for details.
 \item{\joption{caption=true|false}}: If \joption{false}, the \textsf{caption} package is not loaded. This affects the caption layout.
 \item{\joption{covington=true|false}}: If \joption{false}, \textsf{covington} is not loaded. Covington is used for numbered examples.
 \item{\joption{microtype=true|false}}: If \joption{false}, \textsf{microtype} is not loaded.
 \item{\joption{ref=true|false}}: If \joption{false}, \textsf{prettyref} is not loaded and the string (re)definitions
        of the class (concerning verbose cross references) are omitted.
  \item{\joption{tcolorbox=true|false}}: If \joption{false}, \textsf{tcolorbox} is not loaded. Not that in this case, the framed
       boxes are not available.
\end{description}


\subsection{Draft mode}\label{draft}

The option \joption{draftmark=true|false} allows you to mark your document as a draft, which is indicated by a watermark (including the current date). This might be useful when sharing preliminary versions with your supervisor.


\section{Titling and author data}

In this section, it is explained how to input the title and author information that should be
printed on the poster. This information is automatically set, given that you have specified the following data in the preamble.

\subsection{Author-related data and poster title}

\begin{description}
 \setlength\itemsep{0pt}
  \item{\jcsmacro{author[<in footline>]\{<in title>\}}}: Name of the poster's author as it appears in the title and in the footline.
      If you want to omit either of them, just pass an empty argument.
  \item{\jcsmacro{title\{<title>\}}}: Poster title.
  \item{\jcsmacro{subtitle\{<subtitle>\}}}: Poster subtitle.
  \item{\jcsmacro{date\{<date>\}}}: Date of the poster presentation (printed in the foot line).
  \item{\jcsmacro{department\{<name>\}}}: Name of the department (\emph{Department of Linguistics} or \emph{Institut f�r Sprachwissenschaft},
         dependent on the \texttt{cd} option, is preset).
\end{description}
%
Other contact data should be presented in a box on the poster.

\subsection{Presentation-related data}

\begin{description}
 \setlength\itemsep{0pt}
 \item{\jcsmacro{eventtitle\{<title>\}}}: Title of the event where the poster is presented.
 \item{\jcsmacro{eventlocation\{<location>\}}}: Location of the event where the poster is presented.
 \item{\jcsmacro{eventdate\{<date>\}}}: Date of the event where the poster is presented (this might cover
      more days than the poster presentation date, or be a whole term).
 \item{\jcsmacro{eventlogo\{<logo data>\}}}: Logo of the event. This can be an arbitrary text or a command that loads
      a graphic (e.g., \verb|\includegraphics| or \verb|\pgfuseimage|). You need to take care that the logo is properly
      scaled to fit the poster headline.
\end{description}
%
The presentation-related data is printed in the poster headline, right to the University logo. By default, if a logo
is specified, it is printed on the first line, then a line with the event title follows, a third line holds location and date.

This can be modified by changing the separators between the respective elements, which are by default:
\begin{itemize}
	\setlength\itemsep{0pt}
	\item \verb|\def\uvpt@logo@event@sep{\\[.5em]}| -- separation of logo and event title (line break by default)
	\item \verb|\def\uvpt@event@location@sep{\\[.5em]}| -- separation of event title and location (line break by default)
	\item \verb|\def\uvpt@location@date@sep{ $\cdot$ }| -- separation of event location and date (centered dot by default)
\end{itemize}
%
Note that you need to embrace redefinitions in \verb|\makeatletter| and \verb|\makeatother|.

\section{Poster content}

Basically, the poster is nothing more than a \textsf{beamer} slide (of unusual size) and mainly for printout
rather than projection.

Since the class builds on \textsf{beamer}, principally all elements provided by \textsf{beamer} can be used
(although overlay elements do not make sense in printed output).
In what follows, we describe the elements that are most important to posters.

\subsection{The main frame}

A poster is set up as one single beamer frame without frame title. So each poster body is embraced in
\begin{lstlisting}
\begin{frame}
...
\end{frame}
\end{lstlisting}

\subsection{Columns}

Within this frame, it is usually advisable to use multiple columns due to the large size of posters,
since text lines should not be too long to be easily readable (40 characters is a good threshold).
Given this, landscape posters usually have more columns than portrait ones.

Columns can be set up by using \textsf{beamer}'s column framework:
\begin{lstlisting}[moretexcs={[1]{column}}]
\begin{columns}[t, totalwidth=\textwidth]
	% First column, 48% text width
	\column{.48\textwidth}
	<content>
	% Second column, 48% text width
	\column{.48\textwidth}
	<content>
	...
\end{columns}
\end{lstlisting}
%
The option, \texttt{t}, thereby assures that the columns are top-aligned within the centered main frame,
\verb|totalwidth=\textwidth| is needed to have the proper
column span (without this option, the columns will stick into the left margin).

You can freely use columns on your poster, having single-column parts next to two or more columns in turn.
Please refer to the templates for an example.

\subsection{Boxes}

It is established practice to use boxes to present poster sections. This class provides two types of boxes,
each in three colors, blue, red, and green (from the color palette of University of Vienna corporate
design).

The first box type employs a simplistic design that just colors the headline and indents the box content (this
draws on beamer color boxes), the second one draws a colored frame around the box content and also
uses a pale background color (this draws on tcolorbox and is only available if the option \texttt{tcolorbox}
is not set to false).

The first-type boxes can be accessed via
\begin{itemize}
	\setlength\itemsep{0pt}
	\item \verb|\begin{bluebox}{Box Title}...\end{bluebox}|
	\item \verb|\begin{redbox}{Box Title}...\end{redbox}|
	\item \verb|\begin{greenbox}{Box Title}...\end{greenbox}|
\end{itemize}
%
The second type is available via
\begin{itemize}
	\setlength\itemsep{0pt}
	\item \verb|\begin{blueframedbox}{Box Title}...\end{blueframedbox}|
	\item \verb|\begin{redframedbox}{Box Title}...\end{redframedbox}|
	\item \verb|\begin{greenframedbox}{Box Title}...\end{greenframedbox}|
\end{itemize}
%
Our advise is that you use colors carefully. Do not just use them to make
your poster look colorful, but assign them with semantic meaning, e.\,g.,
by using blue for general information, green for examples, and red for
particularly important information.

If you need more fancy boxes, check out the mighty possibilities of \textsf{tcolorbox} \cite{tcolorbox}.
If you need to widen the color range, you can do so by defining your own \texttt{beamercolorbox} or \texttt{tcolorbox} (see
\textsf{beamer} and \textsf{tcolorbox} manuals for details).
We advise you to align with the color palette of the corporate design.%
\footnote{\raggedright\url{https://communications.univie.ac.at/fileadmin/user_upload/d_oeffentlichkeitsarbeit/Dokumente/UniversitaetWien_CD_Manual_Mai_2022_interaktiv.pdf}.}

\subsection{Semantic markup}

The class defines some basic semantic markup common in linguistics:

\begin{description}
 \setlength\itemsep{0pt}
 \item{\jcsmacro{Expression\{<text>\}}}: To mark expressions (object language). Typeset in \emph{italics}.
 \item{\jcsmacro{Concept\{<text>\}}}: To mark concepts. Typeset in \textsc{small capitals}.
 \item{\jcsmacro{Meaning\{<text>\}}}: To mark meaning. Typeset in `single quotation marks'.
\end{description}
You can redefine each of these commands, if needed, like this:
\begin{lstlisting}[language={[LaTeX]TeX},basicstyle={\small\ttfamily},frame=single,morekeywords={enquote}]
\renewcommand*\Expression[1]{\textit{#1}}
\renewcommand*\Concept[1]{\textsc{#1}}
\renewcommand*\Meaning[1]{\enquote*{#1}}
\end{lstlisting}

\subsection{Bigger text}

Since texts on posters might be larger than usual, the class provides three size layer beyond \verb|\Huge|, namely
(in increasing size): \verb|\veryHuge|, \verb|\VeryHuge|, and \verb|\VERYHuge|. Like the legacy size commands, these
are switches.

\subsection{Linguistic examples and glosses}

The class automatically loads the \textsf{covington} package which provides macros for examples and glosses.
Please refer to the \textsf{covington} manual \cite{covington} for details.

\subsection{Bibliography}\label{bibliography}

You can use the usual \LaTeX\ means -- the \textsf{bibliography} environment, Bib\TeX\ or \textsf{biblatex} --
to generate a bibliography and references on your poster. This class assumes you use \textsf{biblatex}, which
is preloaded with the style common in Applied Linguistics at the University of Vienna. See sec.~\ref{sec:bib-different}
on how to opt out of this preselection.

\subsubsection{Default bibliography style (\emph{Unified Style for Linguistics})}

By default, the \uvlt\ class loads a bibliography style which matches the conventions that are recommended by the Applied Linguistics staff of the department.\footnote{See \url{http://www.spitzmueller.org/docs/Zitierkonventionen.pdf}} These conventions draw on the
\emph{Unified Style Sheet for Linguistics} of the LSA (\emph{Linguistic Society of America}), a style that is also quite common in General Linguistics nowadays.
In order to conform to this style, the \uvlt\ class uses the \textsf{biblatex} package with the \textsf{univie-ling} style that is included in the \uvlt\ package.

If you are in Applied Linguistics, using the default style is highly recommended. The style recommended until 2017, namely APA/DGPs, is also still supported, but its use is no longer encouraged; see sec.~\ref{sec:apa-usage} for details. If you want/need to use a different style, please refer to section~\ref{sec:bib-different} for instructions.

\subsubsection{Using APA/DGPs style}\label{sec:apa-usage}

Until 2017, rather than the Unified Style, the Applied Linguistics staff recommended conventions that drew on the citation style guide of the APA
(\emph{American Psychological Association}) and its adaptation for German by the DGPs (\emph{Deutsche Gesellschaft f�r Psychologie}).

For backwards compatibility reasons, this style is still supported (though not recommended). You can enable it with the package option \joption{apa=true}.

If you want to use APA/DGPs style, consider the following caveats.

\begin{itemize}
	\item For full conformance with the APA/DGPs conventions (particularly with regard to the rather tricky handling of ``and'' vs. ``\&'' in- and outside of parentheses), it is mandatory that you adequately use the respective \textsf{biblatex}(\textsf{-apa}) citation commands: Use \jfcsmacro{textcite} for all inline citations and \jfcsmacro{parencite} for all parenthesized citations (instead of manually wrapping \jfcsmacro{cite} in parentheses). If you cannot avoid manually set parentheses that contain citations, use \jfcsmacro{nptextcite} (a \textsf{biblatex-apa}-specific command) inside them.\footnote{ Please refer to \cite{bibltx} and \cite{apabibltx} for detailed instructions.} For quotations, it is recommended to use the quotation macros\slash environments provided by the \textsf{csquotes} package (which is preloaded by \uvlt\ anyway);
	the \uvlt\ class assures that citations are correct if you use the optional arguments of
	those commands\slash macros in order to insert references.
	
	\item The \textsf{biblatex-apa} style automatically lowercases English titles. This conforms to the APA (and DGPs) conventions, which favour ``sentence casing'' over ``title casing''. English titles, from \textsf{biblatex}'s point of view, are titles of bibliographic entries that are either coded as \joption{english} via the \joption{LangID} entry field or that have no \joption{LangID} coding but appear in an English document (i.\,e., a document with main language English). Consequently, if the document's main language is English, all non-English entries need to be linguistically coded (via \joption{LangID}) in order to prevent erroneous lowercasing, since \textsf{biblatex} assumes that non-identified entries use the main language (hence, such a classification is also important for correct hyphenation of the entries).\\
	Note that up to \textsf{biblatex} 3.3, the document language was not taken into account by the lowercasing automatism and all non-classified entries were treated like English entries (and thus lowercased), notwithstanding the main language; therefore, any entry needed to be coded. Even if this misbehaviour is fixed as of \textsf{biblatex} 3.4, it is still advisable to systematically set the proper \joption{LangID}, since this is a prerequisite for a correct multilingual bibliography.
	
	\item The lowercasing automatism described above cannot deal properly with manual punctuation inside titles. Hence, a title such as \texttt{Maintitle.~A subtitle} will come out as \emph{Main title. a subtitle}. There are several ways to avoid that. The most proper one is to use the \textsf{title} and \textsf{subtitle} fields rather than adding everything to \textsf{title}. Alternatively, everything that is nested inside braces will not get lowercased, i.\,e. \texttt{Maintitle.~\{A\} subtitle} will produce the correct result. This trick is also needed for names and other elements that should not get lowercased (\texttt{Introduction to \{Germanic\} linguistics}). However, please do not configure your BibTeX editor to generally embrace titles (this is a feature provided by many editors) since this will prevent \textsf{biblatex-apa} from lowercasing at places where it should be done.
	
	\item The \textsf{biblatex-apa} style requires that you use \textsf{biber} as a
	bibliography processor instead of \textsf{bibtex} (the program). See \cite{biber}
	for details.
\end{itemize}

\subsubsection{Using a different style}\label{sec:bib-different}

If you do not want or are not supposed to use neither the default Unified nor the APA/DGPs style, you can disable automatic \textsf{biblatex} loading via the class option \joption{biblatex=false} (see sec.~\ref{packageloading}). In this case, you will need to load
your own style manually, by entering the respective \textsf{biblatex} or Bib\TeX\ commands.

One case where you need to do that is if you prefer classic Bib\TeX\ over \textsf{biblatex}.
If you want to follow the Applied Linguistics conventions, but prefer classic Bib\TeX\ over \textsf{biblatex}, a Bib\TeX\ style file \textsf{unified.bst} that implements the \emph{Unified Style Sheet for Linguistics} is available on the Internet.\footnote{\url{http://celxj.org/downloads/unified.bst}} Note, though, that this package does not have specific support for German, so it is only really suitable if you write in English. Thus, if you want to follow the Applied Linguistics conventions, it is strongly recommended that you use \textsf{biblatex} with the preloaded \textsf{univie-ling} style.


\subsubsection{Heading}

On posters, you sometimes want a different heading over the references (e.\,g., \emph{Selected Literature}). This can be easily done via the macro
\jcsmacro{Bibheading\{<title>\}}
which works with biblatex and normal bibliography environment.


\section{Further instructions}

\subsection{Commands and environments}

Since the class draws on \textsf{beamer}, you can use all commands and environments provided by \textsf{beamer} in order to structure and typeset your document.
Please refer to the comprehensive beamer manual \cite{beamer} for information.

Please also refer to the template files included in the package for some further usage instructions and hints.

\subsection{\LyX\ layouts and templates}

A layout for \LyX\footnote{See \url{https://www.lyx.org}.}\ can be retrieved from \url{https://github.com/jspitz/univie-ling/raw/master/lyx/layouts/univie-ling-poster.layout}.

Templates are provided as well:
\begin{flushleft}
\begin{itemize}
	\item English template:\\
	      \url{https://github.com/jspitz/univie-ling/raw/master/lyx/templates/template-univie-ling-poster-english.lyx}
	\item German template:\\
	      \url{https://github.com/jspitz/univie-ling/raw/master/lyx/templates/template-univie-ling-poster-deutsch.lyx}
\end{itemize}
\end{flushleft}

\section{Release History}

\begin{description}
  \item 2022/12/06 (forthcoming)
	\begin{itemize}
		\item Initial release.
	\end{itemize}
\end{description}

\begin{thebibliography}{1}

\bibitem{covington} Covington, Michael A. and Spitzm�ller, J�rgen:
\emph{The covington Package. Macros for Linguistics}. September 7, 2018.
\url{http://www.ctan.org/pkg/covington}.

\bibitem{apabibltx} Kime, Philip:
\emph{APA Bib\LaTeX\ style. Citation and References macros for Bib\LaTeX}. March 3, 2016.
\url{http://www.ctan.org/pkg/biblatex-apa}.

\bibitem{beamer} Till Tantau, Joseph Wright and Vedran Mileti\'{c} (2022): \emph{beamer --
A \LaTeX\ class for producing presentations and slides}. URL: \url{http://www.ctan.org/pkg/beamer}.

\bibitem{beamerposter} 	Thomas Deselaers and Philippe Dreuw (2018): \emph{beamerposter -- Extend beamer
and a0poster for custom sized posters}. URL: \url{https://ctan.org/pkg/beamerposter}.

\bibitem{biber} Kime, Philip and Charette, Fran\c{c}ois:
\emph{Biber. A backend bibliography processor for biblatex}. March 6, 2016.
\url{http://www.ctan.org/pkg/biber}.

\bibitem{bibltx} Lehman, Philipp (with Audrey Boruvka, Philip Kime and Joseph Wright):
\emph{The biblatex Package. Programmable Bibliographies	and Citations}. March 3, 2016.
\url{http://www.ctan.org/pkg/biblatex}.

\bibitem{tcolorbox} Thomas F. Sturm (2022): \emph{tcolorbox -- Coloured boxes, for LaTeX
	examples and theorems, etc}. \url{https://ctan.org/pkg/tcolorbox}.

\end{thebibliography} 

\end{document}
