\documentclass[english]{article}

\usepackage[osf]{libertine}
\usepackage[scaled=0.7]{beramono}
\usepackage[T1]{fontenc}
\usepackage[latin9]{inputenc}
\usepackage[hyphens]{url}
\usepackage[pdfusetitle,
 bookmarks=true,bookmarksnumbered=false,bookmarksopen=false,
 breaklinks=false,pdfborder={0 0 0},backref=false,colorlinks=false]
 {hyperref}

% Tweak the TOC (make it more compact)
\usepackage{tocloft}
\setlength{\cftaftertoctitleskip}{6pt}
\setlength{\cftbeforesecskip}{3pt}
\setlength{\cftbeforesubsecskip}{-1.5pt}
\renewcommand{\cfttoctitlefont}{\normalsize\bfseries}
\renewcommand{\cftsecfont}{\small\bfseries}
\renewcommand{\cftsecpagefont}{\small\bfseries}
\renewcommand{\cftsubsecfont}{\small}
\renewcommand{\cftsubsecpagefont}{\small}

% markup
\newcommand*\jmacro[1]{\textbf{\texttt{#1}}}
\newcommand*\jcsmacro[1]{\jmacro{\textbackslash{#1}}}
\newcommand*\joption[1]{\textbf{\texttt{#1}}}
\newcommand*\jfmacro[1]{\texttt{#1}}
\newcommand*\jfcsmacro[1]{\jfmacro{\textbackslash{#1}}}

% macros
\newcommand*\uvlt{\textsf{univie-ling-thesis}}
\providecommand{\LyX}{\texorpdfstring{L\kern-.1667em\lower.25em\hbox{Y}\kern-.125emX\@}{LyX}}

% improve layout
\tolerance 1414
\hbadness 1414
\emergencystretch 1.5em
\hfuzz 0.3pt
\widowpenalty = 10000
\vfuzz \hfuzz
\raggedbottom

\usepackage{microtype}

\usepackage{babel}

\usepackage{listings}
\lstset{language={[LaTeX]TeX},
        basicstyle={\small\ttfamily},
        frame=single}

\setcounter{tocdepth}{2}

\begin{document}

\title{The \uvlt\ class}

\author{\texorpdfstring{J�rgen Spitzm�ller%
		\thanks{Please report issues via \protect\url{https://github.com/jspitz/univie-ling}.}}{J�rgen Spitzm�ller}}

\date{Version 2.9, 2024/09/28}

\maketitle

\begin{abstract}
\noindent The \uvlt\ class provides a \LaTeXe\ class suitable for Bachelor's, Master's, Diploma and Doctoral theses in
(Applied) Linguistics at the Department of Linguistics, University of Vienna.
The class implements some standards required for those theses (such as a suitable title page) and pre-loads some packages
that are considered useful in the context of Applied Linguistics. The class might also be used for General and Historical
Linguistics as well as for other fields of study at Vienna University. In this case, however, some settings might have to
be adjusted. This manual documents the class as well as the configuration possibilities.
\end{abstract}

\tableofcontents

\section{Aims and scope}

The \uvlt\ class has been written mainly with my own field in mind: Applied Linguistics. Therefore, the defaults are closely tied to
the requirements in this field. This particularly concerns the preloaded bibliography style, which conforms to the standards that are proposed by the Viennese Applied Linguistics staff (see sec.~\ref{bibliography}). Furthermore, some frequently used
packages (such as \textsf{covington}) are pre-loaded. As documented later, however, you can disable this (and other) default(s), use
a bibliography style of your choice and load alternative packages.

The design matches as closely as necessary the standards set up by the university. This particularly concerns the
title page, which follows the recommendations specified by the \emph{StudienServiceCenter}.\footnote{\raggedright%
See \url{http://ssc-philkultur.univie.ac.at/studium/masterstudien/abgabe-der-masterarbeit} for master's theses,
\url{http://ssc-philkultur.univie.ac.at/studium/doktoratsstudium-neu-792-xxx/dissertation} for doctoral theses <25.\,01.\,2017>.}

\section{Requirements of \uvlt}\label{sec:req-jslp}

The following class and packages are required and loaded by \uvlt:
\begin{itemize}
 \setlength\itemsep{0pt}
 \item \textsf{scrreprt}: KOMA-Script report class (base class).
 \item \textsf{array}: Tabular extensions.
 \item \textsf{csquotes}: Context sensitive quotations.
 \item \textsf{geometry}: Page layout settings.
 \item \textsf{graphicx}: Graphic support.
 \item \textsf{isodate}: Format URL dates.
 \item \textsf{l3keys}: Key-value interface for class options.
 \item \textsf{scrlayer-scrpage}: Page header settings.
 \item \textsf{setspace}: Line spacing adjustments.
 \item \textsf{translator}: Localization machinery.
 \item \textsf{url}: Support for typesetting URLs.
\end{itemize}
The following packages are required for specific features and loaded by default. However, the loading can be individually and generally omitted (see sec.~\ref{coptions}):
\begin{itemize}
 \setlength\itemsep{0pt}
 \item \textsf{mathptmx}: Default serif font (\emph{Palatino}).
 \item \textsf{sourcesanspro}: Default sans serif font (\emph{Source Sans Pro}).
 \item \textsf{sourcecodepro}: Default monospaced font (\emph{Source Code Pro}).
 \item \textsf{biblatex}: Contemporary bibliography support.
 \item \textsf{caption}: Caption layout adjustments.
 \item \textsf{covington}: Support for linguistic examples\slash glosses.
 \item \textsf{fontenc}: Set the font encoding for PostScript fonts. Loaded with option \joption{T1}.
 \item \textsf{microtype}: Micro-typographic adjustments.
 \item \textsf{prettyref}: Verbose cross-references.
 \item \textsf{varioref}: Context-sensitive cross references.
\end{itemize}
The following packages are required for optional features (not used by default):
\begin{itemize}
 \setlength\itemsep{0pt}
 \item \textsf{biblatex-apa}: APA style for \textsf{biblatex}.
 \item \textsf{draftwatermark}: Create a draft mark.
 \item \textsf{fontspec}: Load OpenType fonts (with LuaTeX or XeTeX).
 \item \textsf{polyglossia}: Multi-language and script support.
 \item \textsf{pdfa}: Create PDF/A compliant files.
\end{itemize}

\section{Fonts}\label{fonts}

The class uses, by default, PostScript (a.\,k.\,a. Type\,1) fonts and thus requires classic (PDF)LaTeX. Optionally, however, you can also use OpenType fonts via the \textsf{fontspec} package and the XeTeX or LuaTeX engine instead. In order to do this, use the class option \joption{fonts=otf} (see sec.~\ref{coptions} for details).

In both cases, the class uses by default \emph{Palatino} as a serif font (or the clone \emph{TeX Gyre Pagella} with OpenType, for that matter),
\emph{Source Sans Pro} as a sans serif font, and \emph{Source Code Pro} as a monospaced (typewriter) font.

As to font selection: \emph{Source Sans Pro} is used in the official title page template which is nowadays auto-generated by the theses server,
so I used that.
A serif font recommendation is not given. The corporate design manual of the university suggests \emph{Source Serif Pro}, but this font is
too heavy for a thesis. A somewhat standard for academic papers and theses is \emph{Times New Roman}, but I think this runs too small for
the text block size we have (as the name suggests, this font has been developed for the small columns of newspapers), and it is also pretty overused.
The selected font, \emph{Palatino}, in contrast, suits well to wider text lines, it is well supported in \LaTeX, and it fits with \emph{Source Sans Pro}.
The monospaced font, \emph{Source Code Pro}, is a good companion to this pair.

If you (or your instructor) prefer(s) \emph{Times New Roman} over \emph{Palatino} nonetheless, write to your preamble
\begin{lstlisting}[language={[LaTeX]TeX},basicstyle={\small\ttfamily},frame=single]
\usepackage{mathptmx}
\end{lstlisting}
if you use PostScriptFonts, or
\begin{lstlisting}[language={[LaTeX]TeX},basicstyle={\small\ttfamily},frame=single,morekeywords={setmainfont}]
\setmainfont{Times New Roman}
\end{lstlisting}
if you use \joption{fonts=otf}, respectively.

If you do neither like \emph{Palatino} nor \emph{Times}: A recommendable serif font (and actually the former `professional' house font of Vienna university)
is \emph{MinionPro}, supported by the excellent \textsf{FontPro} package.%
\footnote{\url{https://github.com/sebschub/FontPro} <25.\,01.\,2017>.} However, some effort is needed to install the package and fonts.
Please refer to the package's documentation in case you are interested. A more accessible alternative, with a similar style than \emph{MinionPro},
is the \emph{CrimsonPro} font, which is available in modern \LaTeX\ distributions via the \textsf{cochineal} package. Another good option is the font
this manual is typeset in, \emph{Linux Libertine} (via the \textsf{libertine} package).

Note that by default, with PostScript fonts, \uvlt\ also loads the \textsf{fontenc} package with T1 font encoding, but this can be customized
(see sec.~\ref{coptions} for details).

If you want (or need) to load all fonts and font encodings manually, you can switch off all automatic loading of fonts and font encodings by
the class option \joption{fonts=none} (see sec.~\ref{coptions}).


\section{Class Options}\label{coptions}

The \uvlt\ class provides a range of key=value type options to control the font handling, package loading and some specific behavior.
These are documented in this section.

\subsection{Font selection}

As elaborated above, the package supports PostScript fonts (via LaTeX and PDFLaTeX) as well as OpenType fonts (via XeTeX and LuaTeX).
PostScript is the traditional LaTeX font format. Specific LaTeX packages and metrics files are needed to use the fonts (but all fonts
needed to use this class should be included in your LaTeX distribution and thus ready to use). OpenType fonts, by contrast, are
taken directly from the operating system. They usually provide a wider range of glyphs, which might be a crucial factor for a linguistic
thesis. However, they can only be used by newer, partly still experimental TeX engines such as XeTeX and LuaTeX.

The class provides the following option to set the font handling:
\begin{description}
 \setlength\itemsep{0pt}
 \item{\joption{fonts=ps|otf|none}}: if \joption{ps} is selected, PostScript fonts are used (this is the default and
       the correct choice if you use LaTeX or PDFLaTeX); if \joption{otf} is selected, OpenType fonts are used, the class
       loads the \textsf{fontspec} package, sets \emph{Palatino} as main font, \emph{Source Sans Pro} as sans serif font,
       and \emph{Source Code Pro} as monospaced font (this is
       the correct choice if you use XeTeX or LuaTeX; make sure you have the respective fonts installed on your system);
       if \joption{none} is selected, finally, the class will not load any font package at all, and neither 
       \textsf{fontenc} (this choice is useful if you want to control the font handling completely yourself).
\end{description}
%
With PostScript fonts, \uvlt\ also loads the \textsf{fontenc} package with T1 font encoding, which is suitable for
most Western European (and some Eastern European) writing systems. In order to load different, or more, encodings, the class option
\begin{description}
	\setlength\itemsep{0pt}
	\item{\joption{fontenc=<encoding(s)>}} can be used (e.\,g., \joption{fontenc=\{T1,X2\}}).
	With \joption{fontenc=none}, the loading of the \textsf{fontenc} package can be prevented. The package is also not loaded with
	\joption{fonts=none}.
\end{description}

\subsection{Polyglossia}

If you need \textsf{polyglossia} rather than \textsf{babel} for language support, please do not use the package yourself, but
rather use the package option \joption{polyglossia=true}. This assures correct loading order. This also presets \joption{fonts=otf}.

\subsection{Package loading}\label{packageloading}

Most of the extra features provided by the class can be switched off. This might be useful if you do not need the respective feature anyway,
and crucial if you need an alternative package that conflicts with one of the preloaded package.

All following options are \joption{true} by default. They can be switched off one-by-one via the value \joption{false}, or altogether, by means of the special option \joption{all=false}. You can also switch selected packages on\slash off again after this
option (e.\,g., \joption{all=false,microtype=true} will switch off all packages except \textsf{microtype}).

\begin{description}
 \setlength\itemsep{0pt}
 \item{\joption{apa=true|false}}: If \joption{true}, the \textsf{biblatex-apa} style is used when \textsf{biblatex} is loaded. 
        By default, the included \textsf{univie-ling} style is loaded, instead. See sec.~\ref{bibliography} for details.
 \item{\joption{biblatex=true|false}}: If \joption{false}, \textsf{biblatex} is not loaded. This is useful if you prefer Bib\TeX\
        over \textsf{biblatex}, but also if you neither want to use the preloaded \textsf{univie-ling} style nor the
        alternative \textsf{biblatex-apa} style (i.\,e., if you want to load \textsf{biblatex}  manually with different options).
        See sec.~\ref{bibliography} for details.
 \item{\joption{caption=true|false}}: If \joption{false}, the \textsf{caption} package is not loaded. This affects the caption layout.
 \item{\joption{covington=true|false}}: If \joption{false}, \textsf{covington} is not loaded. Covington is used for numbered examples.
 \item{\joption{microtype=true|false}}: If \joption{false}, \textsf{microtype} is not loaded.
 \item{\joption{ref=true|false}}: If \joption{false}, both \textsf{prettyref} and \textsf{varioref} are not loaded and the string (re)definitons
        of the class (concerning verbose cross references) are omitted.
\end{description}


\subsection{Draft mode}\label{draft}

The option \joption{draftmark=true|false|firstpage} allows you to mark your document as a draft, which is indicated by a watermark (including the current date). This might be useful when sharing preliminary versions with your supervisor.
With \joption{draftmark=true}, this mark is printed on top of each page.  With \joption{draftmark=firstpage}, the draft mark appears on the title page only.


\subsection{Title page}\label{titlepage}

As of 2024, MA and PhD theses must be submitted to the \textsf{u:space} platform without a title page (the title page will then be generated by the platform from the data
you fill in there).

Use the option \joption{titlepage=false} to suppress the title page for this purpose. The pagination will remain the same as with titlepage (i.\,e., the document starts at page 2).

\subsection{Further options}

\begin{description}
 \setlength\itemsep{0pt}
 \item{\joption{fdegree=true|false}}: Prefer feminine forms for the targeted degree on the title page (such as \emph{Magistra}, \emph{Doktorin}). Default: \joption{false}.
 \item{\joption{pdfa=true|false}}: Generate PDF/A compliant output (see sec.~\ref{sec:pdfa} for details). Default: \joption{false}.
\end{description}
The class builds on \textsf{scrreprt} (KOMA report), which provides many more options to tweak the appearance of your document. You can use
all these options via the \jfcsmacro{KOMAoptions} macro. Please refer to the \textsf{KOMA-Script} manual \cite{koma} for details.

\section{General settings}

In this section, it is explained how you can enter some general settings, particular the information that must be given on the title page. 
The title page, following the university guidelines, is automatically set up by the \jcsmacro{maketitle} command, given that you have specified
the following data in the preamble.

\subsection{Author-related data}

\begin{description}
 \setlength\itemsep{0pt}
 \item{\jcsmacro{author\{<name>\}}}: Name of the thesis author. Separate multiple authors by \jcsmacro{and}.
 \item{\jcsmacro{studienkennzahl\{<code>\}}}: The degree programme code  (\emph{Studienkennzahl}) as it appears on
 the student record sheet, e.\,g. \emph{A\,792\,327}.
 \item{\jcsmacro{studienrichtung\{<field of study>\}}}: Your degree programme (\emph{Studienrichtung}) or field of study (\emph{Dissertationsgebiet}) as it appears on
 the student record sheet, e.\,g. \emph{Sprachwissenschaft}.
\end{description}


\subsection{Thesis-related data}

\begin{description}
 \setlength\itemsep{0pt}
 \item{\jcsmacro{thesistype\{<type>\}}}: The type of your thesis. Possible values include \joption{magister} (Magisterarbeit), \joption{diplom} (Diploma Thesis),
          \joption{bachelor} (Bachelor's Thesis), \joption{master} (Master's Thesis) and \joption{diss} (Doctoral Thesis).
 \item{\jcsmacro{title\{<title>\}}}: Title of the thesis.
 \item{\jcsmacro{subtitle\{<subtitle>\}}}: Subtitle of the thesis.
 \item{\jcsmacro{volume\{<current>\}\{<total>\}}}: If your thesis consists of multiple volumes, specifiy the current volume as well as the total number of volumes.
 \item{\jcsmacro{supervisor\{<name>\}}}: Title and name of your (main) supervisor.
 \item{\jcsmacro{cosupervisor\{<name>\}}}: Title and name of your co-supervisor.
\end{description}
The suitable degree (\emph{Angestrebter akademischer Grad} in German) is automatically set by the \jcsmacro{thesistype} command, but you can override it with the optional command \jcsmacro{degree\{<custom degree>\}}.
Note that female forms of degrees, where appropriate, are  used if you use the class option \joption{fdegree=true} (see sec.~\ref{coptions}).


\section{Declaration}\label{decl}

It is possible to automatically generate a page with a declaration where you declare and sign that you follow research ethics\slash anti-plagiarism rules
(\emph{Selbst\"andigkeitserkl\"arung}) by means of the command

\begin{lstlisting}[language={[LaTeX]TeX},basicstyle={\small\ttfamily},frame=single,morekeywords={makedeclaration}]
	\makedeclaration
\end{lstlisting}
%
Such a declaration is needed for BA theses.

\section{Semantic markup}\label{markup}

The class defines some basic semantic markup common in linguistics:

\begin{description}
 \setlength\itemsep{0pt}
 \item{\jcsmacro{Expression\{<text>\}}}: To mark expressions (object language). Typeset in \emph{italics}.
 \item{\jcsmacro{Concept\{<text>\}}}: To mark concepts. Typeset in \textsc{small capitals}.
 \item{\jcsmacro{Meaning\{<text>\}}}: To mark meaning. Typeset in `single quotation marks'.
\end{description}
You can redefine each of these commands, if needed, like this:
\begin{lstlisting}[language={[LaTeX]TeX},basicstyle={\small\ttfamily},frame=single,morekeywords={enquote}]
\renewcommand*\Expression[1]{\textit{#1}}
\renewcommand*\Concept[1]{\textsc{#1}}
\renewcommand*\Meaning[1]{\enquote*{#1}}
\end{lstlisting}

\section{Hyperlinks}

The stylesheet requires you to always give last access dates to web links (URLs). To make this easier,
\uvlt\ provides a number of convenience macros. To begin with, you can easily enter URLs with the last date you
accessed the site with the function
\begin{lstlisting}[language={[LaTeX]TeX},basicstyle={\small\ttfamily},frame=single,morekeywords={weblink}]
\weblink{<URL>}[<date>]
\end{lstlisting}
%
This will format the date in the preferred way (``[last accessed: <date>]''), localized to English or German.
The argument can be given in the form YYYY-MM-DD, DD/MM/YYYY, or DD.MM.YYYY, e.g.
\begin{lstlisting}[language={[LaTeX]TeX},basicstyle={\small\ttfamily},frame=single,morekeywords={weblink}]
\weblink{https://linguistics.univie.ac.at}[27/06/2024]
\end{lstlisting}
%
will print: \url{https://linguistics.univie.ac.at} [last accessed: 27/06/2024].
If you do not give the optional date argument, no date is printed unless you have specified a general last access date
by means of the function
\begin{lstlisting}[language={[LaTeX]TeX},basicstyle={\small\ttfamily},frame=single,morekeywords={SetURLDate}]
\SetURLDate{<date>}
\end{lstlisting}
%
which takes the same date format as argument. This function might come in handy if you have the same access date
for all, or many, URLs. And it lets you update all URL access dates with a single change.

For hyperlinks without date, you can also use the legacy command \jcsmacro{url\{<URL>\}}.
To output the date information only, you can use the macro \jcsmacro{urldate\{<date>\}}.

\section{Linguistic examples and glosses}

The class automatically loads the \textsf{covington} package which provides macros for examples and glosses.
Please refer to the \textsf{covington} manual \cite{covington} for details.

\section{Bibliography}\label{bibliography}

\subsection{Default bibliography style (\emph{Unified Style for Linguistics})}

By default, the \uvlt\ class loads a bibliography style which matches the conventions that are recommended by the Applied Linguistics staff of the department.\footnote{See \url{http://www.spitzmueller.org/docs/Zitierkonventionen.pdf}} These conventions draw on the
\emph{Unified Style Sheet for Linguistics} of the LSA (\emph{Linguistic Society of America}), a style that is also quite common in General Linguistics nowadays.
In order to conform to this style, the \uvlt\ class uses the \textsf{biblatex} package with the \textsf{univie-ling} style that is included in the \uvlt\ package.

If you are in Applied Linguistics, using the default style is highly recommended. The style recommended until 2017, namely APA/DGPs, is also still supported, but its use is no longer encouraged; see sec.~\ref{sec:apa-usage} for details. If you want/need to use a different style, please refer to section~\ref{sec:bib-different} for instructions.

\subsection{Using APA/DGPs style}\label{sec:apa-usage}

Until 2017, rather than the Unified Style, the Applied Linguistics staff recommended conventions that drew on the citation style guide of the APA
(\emph{American Psychological Association}) and its adaptation for German by the DGPs (\emph{Deutsche Gesellschaft f�r Psychologie}).

For backwards compatibility reasons, this style is still supported (though not recommended). You can enable it with the package option \joption{apa=true}.

If you want to use APA/DGPs style, consider the following caveats.

\begin{itemize}
	\item For full conformance with the APA/DGPs conventions (particularly with regard to the rather tricky handling of ``and'' vs. ``\&'' in- and outside of parentheses), it is mandatory that you adequately use the respective \textsf{biblatex}(\textsf{-apa}) citation commands: Use \jfcsmacro{textcite} for all inline citations and \jfcsmacro{parencite} for all parenthesized citations (instead of manually wrapping \jfcsmacro{cite} in parentheses). If you cannot avoid manually set parentheses that contain citations, use \jfcsmacro{nptextcite} (a \textsf{biblatex-apa}-specific command) inside them.\footnote{ Please refer to \cite{bibltx} and \cite{apabibltx} for detailed instructions.} For quotations, it is recommended to use the quotation macros\slash environments provided by the \textsf{csquotes} package (which is preloaded by \uvlt\ anyway);
	the \uvlt\ class assures that citations are correct if you use the optional arguments of
	those commands\slash macros in order to insert references.
	
	\item The \textsf{biblatex-apa} style automatically lowercases English titles. This conforms to the APA (and DGPs) conventions, which favour ``sentence casing'' over ``title casing''. English titles, from \textsf{biblatex}'s point of view, are titles of bibliographic entries that are either coded as \joption{english} via the \joption{LangID} entry field or that have no \joption{LangID} coding but appear in an English document (i.\,e., a document with main language English). Consequently, if the document's main language is English, all non-English entries need to be linguistically coded (via \joption{LangID}) in order to prevent erroneous lowercasing, since \textsf{biblatex} assumes that non-identified entries use the main language (hence, such a classification is also important for correct hyphenation of the entries).\\
	Note that up to \textsf{biblatex} 3.3, the document language was not taken into account by the lowercasing automatism and all non-classified entries were treated like English entries (and thus lowercased), notwithstanding the main language; therefore, any entry needed to be coded. Even if this misbehaviour is fixed as of \textsf{biblatex} 3.4, it is still advisable to systematically set the proper \joption{LangID}, since this is a prerequisite for a correct multilingual bibliography.
	
	\item The lowercasing automatism described above cannot deal properly with manual punctuation inside titles. Hence, a title such as \texttt{Maintitle.~A subtitle} will come out as \emph{Main title. a subtitle}. There are several ways to avoid that. The most proper one is to use the \textsf{title} and \textsf{subtitle} fields rather than adding everything to \textsf{title}. Alternatively, everything that is nested inside braces will not get lowercased, i.\,e. \texttt{Maintitle.~\{A\} subtitle} will produce the correct result. This trick is also needed for names and other elements that should not get lowercased (\texttt{Introduction to \{Germanic\} linguistics}). However, please do not configure your BibTeX editor to generally embrace titles (this is a feature provided by many editors) since this will prevent \textsf{biblatex-apa} from lowercasing at places where it should be done.
	
	\item The \textsf{biblatex-apa} style requires that you use \textsf{biber} as a
	bibliography processor instead of \textsf{bibtex} (the program). See \cite{biber}
	for details.
\end{itemize}

\subsection{Using a different style or not using \textsf{biblatex}}\label{sec:bib-different}

If you do not want or are not supposed to use neither the default Unified nor the APA/DGPs style, you can disable automatic \textsf{biblatex} loading via the class option \joption{biblatex=false} (see sec.~\ref{packageloading}). In this case, you will need to load
your own style manually, by entering the respective \textsf{biblatex} or Bib\TeX\ commands.

One case where you need to do that is if you prefer classic Bib\TeX\ over \textsf{biblatex}.
If you want to follow the Applied Linguistics conventions, but prefer classic Bib\TeX\ over \textsf{biblatex}, a (seemingly unmaintained) Bib\TeX\ style file \textsf{unified.bst} that implements the \emph{Unified Style Sheet for Linguistics} is available on the Internet.\footnote{A copy is at \url{https://raw.githubusercontent.com/jspitz/univie-ling/master/3rdparty/unified.bst}.} This style produces English output, a localized version for German,
\textsf{unified-de.bst}, is available as well.\footnote{See \url{https://raw.githubusercontent.com/jspitz/univie-ling/master/3rdparty/unified-de.bst}.}
Both styles require the \textsf{natbib} package to be loaded.
Note, though, that both styles provide only very rudimentary support for the Unified Style Sheet. Thus, if you want to follow the Applied Linguistics conventions
really strictly, it is strongly recommended that you use \textsf{biblatex} with the preloaded \textsf{univie-ling} style.

It is also possible (though not recommended) to use the \texttt{thebibliography} environment if,
for some reason, you do not want to use \textsf{biblatex} nor Bib\TeX. Of course, in this case you are all responsible for the
proper formatting of the entries. If you decide for this, you must load the \textsf{natbib} package and use the
syntax documented in the natbib manual (\cite[sec.~1]{natbib}) for the proper setup of bibitems, e.\,g.,\footnote{%
	Note the missing space before and after the parentheses that mark the year!}

\begin{lstlisting}[language={[LaTeX]TeX},basicstyle={\small\ttfamily},frame=single,morekeywords={bibitem}]
\bibitem[Jones et al.(1990)Jones, Baker, & Williams]{jon90}
	Jones, Baker, & Williams. 1990. ...
\end{lstlisting}
%
(Only) if \textsf{natbib} is loaded, \uvlt\ will take care that the spacing with the \texttt{thebibliography} environment
is correct.

\section{Further instructions}

\subsection{Commands and environments}

Since the class draws on \textsf{scrreprt}, you can use all commands and environments provided by \textsf{KOMA report} in order to structure and typeset your document.
Please refer to the comprehensive KOMA-Script manual \cite{koma} for information.

Please also refer to the template files included in the package for some further usage instructions and hints.

\subsection{Generating PDF/A}\label{sec:pdfa}

In order to submit your thesis to the \emph{Hochschulschriften-Server}, it must use a specific PDF standard aimed at long-time archiving, either PDF/A-1 (with PDF 1.4) or PDF/A-2 (with PDF 1.7).\footnote{\url{https://studienpraeses.univie.ac.at/en/information-on-study-law/academic-theses/submission-of-the-academic-thesis/}}
With PDFLaTeX or LuaTeX, you can achieve PDF/A-1b, PDF/A-2b or PDF/A-2u compliant files by means of the \textsf{pdfx} package.\footnote{XeTeX also works with recent versions of the \textsf{pdfx} package, if you use the command line option \texttt{--shell-escape -output-driver="xdvipdfmx -z 0"} of \texttt{xelatex}. Please refer to the  \textsf{pdfx} manual for details.} If you do not use specific color profiles in
your document and provided that all your graphics follow the requirements of
the PDF/A standard (all fonts embedded, no transparency groups etc.), producing
a PDF/A-1b compliant file is straightforward:

\begin{enumerate}
	\item Use the class option \joption{pdfa=true}. This generates PDF/A-2u and PDF 1.7.

	\item Create a text file called \texttt{<name>.xmpdata} (where <name> is to be replaced by the name of your [master] TeX file or the produced PDF file, respectively), which contains some metadata of your document (author's name, title, keywords, etc.). This file needs to be stored next to your TeX file(s). Example \texttt{*.xmpdata} files are included in the \uvlt\ bundle (and used in the accompanying templates); you can use them as a model and adapt them to your needs.
\end{enumerate}
%
Note that \textsf{pdfx} does not verify whether the produced output really conforms to the standard. You need to use a PDF/A verification tool to ensure
that. If you own Adobe Acrobat Pro, you can use its \emph{preflight} tool for this task. Alternatively, a number of free validation programs and online services are available
on the Internet.

For detailed instructions, please refer to the \textsf{pdfx} package manual \cite{pdfx}.
Furthermore, a step-by-step guide to PDF/A generation with PDFLaTeX and \textsf{pdfx} is available at \url{http://www.mathstat.dal.ca/~selinger/pdfa/}.

\subsection{\LyX\ layouts and templates}

A layout for \LyX\footnote{See \url{https://www.lyx.org}.}\ can be retrieved from \url{https://github.com/jspitz/univie-ling/raw/master/lyx/layouts/univie-ling-thesis.layout}.

Templates are provided as well:
\begin{flushleft}
\begin{itemize}
	\item English template:\\
	\url{https://github.com/jspitz/univie-ling/raw/master/lyx/templates/template-univie-ling-thesis-english.lyx}
	\item German template:\\
	\url{https://github.com/jspitz/univie-ling/raw/master/lyx/templates/template-univie-ling-thesis-deutsch.lyx}
\end{itemize}
\end{flushleft}

\section{Release History}

\begin{itemize}
  \item Forthcoming (v.\,2.10)
	\begin{itemize}
		\item Add alt image text to logos.
		\item Increase PDF/A to A-2u.
	\end{itemize}
  \item 2024/09/28 (v.\,2.9)
	\begin{itemize}
		\item No change to this class.
	\end{itemize}
  \item 2024/09/20 (v.\,2.8)
	\begin{itemize}
		\item Add class option \joption{titlepage=true|false} (see sec.~\ref{titlepage}).
		\item Adapt title page to recent design. This entails a change of sans serif font
		      from Arial to Source Sans Pro.
		\item Use \emph{TeX Gyre Pagella} rather than \emph{Palatino} with \joption{fonts=otf}.
		\item Add basic support for \texttt{thebibliography} environment.
		\item Update documentation on bibliography beyond \textsf{biblatex}.
	\end{itemize}
  \item 2024/07/23 (v.\,2.7)
	\begin{itemize}
		\item Fix spacing after \jfcsmacro{weblink}.
		\item Fix empty argument check in \jfcsmacro{weblink}.
	\end{itemize}
  \item 2024/06/27 (v.\,2.6)
    \begin{itemize}
	   \item Provide \jcsmacro{weblink}, \jcsmacro{SetURLDate} and \jcsmacro{urldate} macros.
       \item Fix running headers in twoside mode.
	\end{itemize}
  \item 2024/05/09 (v.\,2.5)
	\begin{itemize}
		\item No change to this class.
	\end{itemize}
  \item 2023/01/26 (v.\,2.3)
	\begin{itemize}
		\item No change to this class.
	\end{itemize}
  \item 2022/12/06 (v.\,2.2)
	\begin{itemize}
		\item Fix boolean option parsing.
	\end{itemize}
  \item 2022/10/21 (v.\,2.1)
	\begin{itemize}
		\item Fix \joption{polyglossia} option.
	\end{itemize}
  \item 2022/10/02 (v.\,2.0)
	\begin{itemize}
		\item Use \textsf{l3keys} rather than \textsf{xkeyval} for key-value option handling.
		\item Fix some \textsf{varioref} definitions.
		\item Use \textsf{translator} instead of \textsf{translations} for localization.
		\item Various small class cleanups.
	\end{itemize}
  \item 2022/09/08 (v.\,1.20)
	\begin{itemize}
		\item Load \textsf{varioref} AtBeginDocument.
	\end{itemize}
  \item 2022/06/18 (v.\,1.19)
	\begin{itemize}
		\item Add option \joption{fontenc}.
		\item Fix translation of English example string.
		\item Add monospaced font.
	\end{itemize} 
  \item 2022/05/11 (v.\,1.18)
	\begin{itemize}
		\item Fix error with subtitles.
	\end{itemize} 
  \item 2022/02/05 (v.\,1.17)
	\begin{itemize}
		\item Fix option \joption{apa}.
		\item Omit quotation marks when title is empty.
	\end{itemize} 
  \item 2021/11/03 (v.\,1.16)
   \begin{itemize}
 	\item Add \jcsmacro{makedeclaration} (for BA theses). See sec.~\ref{decl}.
 	\item Adjust font size and margins of BA thesis to dept. standards.
 	\item Add option \joption{draftmark}. See sec.~\ref{draft}.
 	\item Fix grouping in \jcsmacro{maketitle}.
 \end{itemize} 
 \item 2021/10/19 (v.\,1.15) No change to this class.
 \item 2021/09/01 (v.\,1.14) No change to this class.
 \item 2020/11/11 (v.\,1.13) No change to this class.
 \item 2020/06/25 (v.\,1.12) No change to this class.
 \item 2020/05/05 (v.\,1.11)
 \begin{itemize}
 	\item New option \joption{polyglossia}.
 	\item New option \joption{pdfa}.
 \end{itemize} 
 \item 2020/05/01 (v.\,1.10) No change to this class.
 \item 2019/01/21 (v.\,1.9) No change to this class.
 \item 2019/01/15 (v.\,1.8) Fix title abbreviations (MA, BA).
 \item 2018/11/07 (v.\,1.7) No change to this class.
 \item 2018/11/04 (v.\,1.6)	Remove \jmacro{subexamples} environment as this is now
       provided by \textsf{covington}.
 \item 2018/09/03 (v.\,1.5)	Introduce \jmacro{subexamples} environment.
 \item 2018/04/26 (v.\,1.4)	Fix full date issue in biblatex bibliography style.
 \item 2018/03/02 (v.\,1.3)	No change to this class.
 \item 2018/02/13 (v.\,1.2)	No change to this class.
 \item 2018/02/11 (v.\,1.1)	No change to this class.
  \item 2018/02/08 (v.\,1.0)
	\begin{itemize}
		\item Switch default bibliography style (from APA to Unified).
		\item  Initial release to CTAN.
	\end{itemize}
  \item 2016/05/05 (v.\,0.9)
	\begin{itemize}
		\item Fix comma after \emph{et al.} with \textsf{biblatex-apa}.
	\end{itemize}
  \item 2016/04/30 (v.\,0.8)
    \begin{itemize}
    	\item Set proper citation command for csquotes' integrated environments.
	    \item Improve templates.
    \end{itemize}
  \item 2016/04/28 (v.\,0.7)
	\begin{itemize}
		\item Fix leading of thesis type on title page.
	\end{itemize}
  \item 2016/03/23 (v.\,0.6)
	\begin{itemize}
		\item Fix the output of German multi-name citations (DGPs guidelines).
		\item Extend documentation of bibliographic features.
	\end{itemize}
  \item 2016/02/08 (v.\,0.5)
  \begin{itemize}
  	\item Fix the PDF logo for PDF/A output.
  	\item Extend documentation of PDF/A production.
  \end{itemize}
  \item 2016/01/25 (v.\,0.4) First (documented) release.
  \begin{itemize}
    \item Title page layout has been adapted to the new (bilingual) standards.
    \item add \joption{bachelor} thesis type.
    \item Possibility to disable some pre-loaded packages.
    \item Use key=value option format.
  \end{itemize}
\end{itemize}

\begin{thebibliography}{1}

\bibitem{covington} Covington, Michael A. and Spitzm�ller, J�rgen:
\emph{The covington Package. Macros for Linguistics}. September 7, 2018.
\url{http://www.ctan.org/pkg/covington}.

\bibitem{natbib} Daly, Patrick W.:
\emph{Natural Sciences Citations and References}. September 13, 2010.
\url{http://www.ctan.org/pkg/natbib}.

\bibitem{apabibltx} Kime, Philip:
\emph{APA Bib\LaTeX\ style. Citation and References macros for Bib\LaTeX}. March 3, 2016.
\url{http://www.ctan.org/pkg/biblatex-apa}.

\bibitem{biber} Kime, Philip and Charette, Fran\c{c}ois:
\emph{Biber. A backend bibliography processor for biblatex}. March 6, 2016.
\url{http://www.ctan.org/pkg/biber}.
	
\bibitem{koma} Kohm, Markus (2015): KOMA-Script. The Guide. URL: \url{http://www.ctan.org/pkg/koma-script}.

\bibitem{bibltx} Lehman, Philipp (with Audrey Boruvka, Philip Kime and Joseph Wright):
\emph{The biblatex Package. Programmable Bibliographies	and Citations}. March 3, 2016.
\url{http://www.ctan.org/pkg/biblatex}.

\bibitem{pdfx} Radhakrishnan, C.\,V. et al. (2016): Generation of PDF/X and PDF/A compliant PDF's with pdf\TeX -- pdfx.sty. \url{http://www.ctan.org/pkg/pdfx}.

\end{thebibliography} 

\end{document}
