% Template for a single WLG article
%
% DOCUMENT CLASS OPTIONS:
%
% titlepage=specialprint Output special print ("Sonderdruck") title page
%                        suitable for single articles
% titlepage=issue        Output title page for whole journal issue
% biblatex=true          Load biblatex (with univie-ling style) internally
% covington=false        Do not load covingtion package (Linguistic macros)
% expertfonts=true       Use MinionPro instead of Crimson
\documentclass[titlepage=specialprint]{univie-ling-wlg}

\usepackage[utf8]{inputenc}

% Supported languages: naustrian, ngerman, english
\RequirePackage[ngerman,naustrian]{babel}

% This is just for the dummy text
% Remove this and the \blind... macros for real documents
\usepackage{blindtext}
% Custom blindtext (Austrian not supported by blindtext package)
\newcommand*{\blindtextAT}{\foreignlanguage{ngerman}{\blindtext}}
\newcommand*{\blindlistAT}{\foreignlanguage{ngerman}{\blindlistlist[3]{itemize}}}

%
% EDITORIAL SETTINGS
%
% Start page number (default: 1)
%\startpage{15}
% Journal issue number and year
\issue{80}{2017}
% Title and subtitle of special issue
%\issuetitle{Titel des Themenhefts}
%\issuesubtitle{Untertitel des Themenhefts}
% Editor(s) of special issue
%\issueeditors{Vorname Name, Vorname Name und Vorname Name}


\begin{document}

% \aff: affiliationaof authors. The starred version marks the corresponding author [m = male] [f = female] [p = plural]
\author[Pan \and Meier]{Peter Pan\aff{Peter Pan, Institut für Experimentellen Bl\"odsinn, 66890 Neverland, pan@ieb.nv}
                        \and Paula Meier\aff*[f]{Paula Meier, Institut für Angewandten Unsinn, 11201 Utopos, meier@iau.edu}}

\title{Aufsatztitel}
\subtitle{Untertitel}

% This lets you set a fixed publication date (by default, \today is used)
% \date{12. Dezember 2021}

\maketitle

\begin{abstract}
% dummy text
\blindtextAT
\end{abstract}

\keywords{Applied Linguistics, Discourse Analysis, Sociolinguistics, Text Analysis}

\section{Abschnitt}

\motto[Quelle]{Ein schickes Motto}

\blindtextAT

\begin{displayquote}
% dummy text
\blindtextAT
\end{displayquote}
% dummy text
\blindtextAT

% dummy list
\blindlistAT

% dummy text
\blindtextAT

% dummy list
\blindlistAT
	
\end{document}
