% Example file for the class ``univie-ling-poster''
\documentclass[naustrian,portrait]{univie-ling-poster}

% Titel und Autor:innenangaben
%
\author[Autor in Fußzeile]{Autor im Titel}
%\department{Alternatives Institut}

\title{Titel des Posters}
\subtitle{Untertitel}
\date{Datum der Präsentation}

% Angaben zur Veranstaltung
%
\eventtitle{Titel der Veranstaltung}
\eventlocation{Ort der Veranstaltung}
\eventdate{Datum der Veranstaltung}
%\eventlogo{}


\begin{document}

% Der gesamte Posterinhalt kommt in einen einzigen Frame
\begin{frame}

\begin{bluebox}{Titel einer blauen Box}
	Text zur Veranschaulichung der Funktion dieses Textelements, ein so
	genannter \enquote{Blindtext}, der nichts weiter tut, als eben zu zeigen,
	wie Text aussieht. Text zur Veranschaulichung der Funktion dieses Textelements, ein so
	genannter \enquote{Blindtext}, der nichts weiter tut, als eben zu zeigen,
	wie Text aussieht.
\end{bluebox}

\begin{greenbox}{Titel einer grünen Box}
	Text zur Veranschaulichung der Funktion dieses Textelements, ein so
	genannter \enquote{Blindtext}, der nichts weiter tut, als eben zu zeigen,
	wie Text aussieht. Text zur Veranschaulichung der Funktion dieses Textelements, ein so
	genannter \enquote{Blindtext}, der nichts weiter tut, als eben zu zeigen,
	wie Text aussieht.
\end{greenbox}


\begin{redbox}{Titel einer roten Box}
	Text zur Veranschaulichung der Funktion dieses Textelements, ein so
	genannter \enquote{Blindtext}, der nichts weiter tut, als eben zu zeigen,
	wie Text aussieht. Text zur Veranschaulichung der Funktion dieses Textelements, ein so
	genannter \enquote{Blindtext}, der nichts weiter tut, als eben zu zeigen,
	wie Text aussieht.
\end{redbox}

\begin{blueframedbox}{Titel einer blauen gerahmten Box}
	Text zur Veranschaulichung der Funktion dieses Textelements, ein so
	genannter \enquote{Blindtext}, der nichts weiter tut, als eben zu zeigen,
	wie Text aussieht. Text zur Veranschaulichung der Funktion dieses Textelements, ein so
	genannter \enquote{Blindtext}, der nichts weiter tut, als eben zu zeigen,
	wie Text aussieht.
	\tcblower
	Gerahmte Boxen haben einen optionalen \enquote*{unteren} Teil
\end{blueframedbox}

\begin{redframedbox}{Titel einer roten gerahmten Box}
	Text zur Veranschaulichung der Funktion dieses Textelements, ein so
	genannter \enquote{Blindtext}, der nichts weiter tut, als eben zu zeigen,
	wie Text aussieht. Text zur Veranschaulichung der Funktion dieses Textelements, ein so
	genannter \enquote{Blindtext}, der nichts weiter tut, als eben zu zeigen,
	wie Text aussieht.
\end{redframedbox}

\begin{greenframedbox}{Titel einer grünen gerahmten Box}
	Text zur Veranschaulichung der Funktion dieses Textelements, ein so
	genannter \enquote{Blindtext}, der nichts weiter tut, als eben zu zeigen,
	wie Text aussieht. Text zur Veranschaulichung der Funktion dieses Textelements, ein so
	genannter \enquote{Blindtext}, der nichts weiter tut, als eben zu zeigen,
	wie Text aussieht.
\end{greenframedbox}

% Spalten über die gesamte Textbreite
\begin{columns}[t, totalwidth=\textwidth]

% Erste Spalte, 48% Textbreite
\column{.48\textwidth}

\begin{redbox}{Titel}
	\begin{enumerate}
	\item Text Text
	\item Text Text
	\item Text Text
	\item Text Text
	\end{enumerate}
\end{redbox}


\begin{bluebox}{Worum es geht}
	\begin{itemize}
	\item Text Text
		\begin{itemize}
		\item Text Text
		\item Text Text
		\end{itemize}
	\item Text Text
	\end{itemize}
\end{bluebox}


% Zweite Spalte, 48% Textbreite
\column{.48\textwidth }

\begin{bluebox}{Titel}
	\begin{itemize}
	\item Text Text
	\item Text Text
	\item Text Text
	\item Text Text
	\end{itemize}
\end{bluebox}

\begin{greenbox}{Kontakt}
   Jane Doe

   u32168@univie.ac.at
\end{greenbox}

\end{columns}% Ende der Spalten

\end{frame}% Ende des Gesamtframes

\end{document}
