% Beispieldatei für die Klasse ``univie-ling-thesis''
% Übersetzen mit pdflatex -- biber -- pdflatex
% Verwende den vom Hochschulschriftenservice verlangten Standard PDF/A1-b
% (siehe http://e-theses.univie.ac.at/elektronische_einreichung.html)
\documentclass[naustrian,pdfa]{univie-ling-thesis}

\usepackage{babel}

% Standardmäßig wird das Biblatex-Paket mit Unified-Stil und
% Biber-Backend geladen.
% Wenn Sie klassisches Bibtex (mit unified.bst) verwenden wollen,
% verwenden Sie die Klassenoption ``biblatex=false''
\bibliography{biblatex-examples}

% Persönliche Angaben
\author{Vorname Nachname, BA MA}
\studienkennzahl{A 792 327}
\studienrichtung{Sprachwissenschaft}

% Mehrere Autor*innen:
%\author{Vorname Nachname, BA  MA \and Vorname Nachname, BA MA}
%\studienkennzahl{A 066 899 \and A 066 899}

% Arbeitsrelevante Angaben
\thesistype{diss}% Mögliche Optionen: magister, diplom, bachelor, master, diss
\title{Titel der Arbeit}
\subtitle{Untertitel}
\volume{1}{5}% Band x von y Bänden
% Der angemessene Grad wird automatisch ausgewählt, aber Sie können
% diese Wahl überschreiben, indem Sie den folgenden Befehl auskommentieren:
% (Beachten Sie, dass weibliche Formen für die Grade verwendet werden, wenn
%  Sie die Klassenoption ``fdegree=true'' verwenden)
%\degree{Doktor der Philosophie (Dr. phil.)}% Angestrebter Grad
\supervisor{Univ.-Prof. Dr. Vorname Nachname}
\cosupervisor{Univ.-Prof. Dr. Vorname Nachname}


\begin{document}

% Die Titelseite
\maketitle

% Selbständigkeitserklärung (für BA-Arbeiten)
% \makedeclaration

\tableofcontents

\chapter{Ein paar Tipps}\label{cha:hinweise}

Verwenden Sie am Besten \textsf{biblatex} für konsistente Literaturverweise im \emph{Unified}-Stil (vgl. \emph{Leitfaden für die Gestaltung von
	schriftlichen Arbeiten und Unterlagen}; März 2018).
Für normale Verweise im Text \verb|\textcite|: \textcite[22]{brandt}, für Verweise in Klammern \verb|\parencite|: \parencite{brandt}.

Für Anführungszeichen und Zitate verwenden Sie am Besten die Befehle des Paketes \emph{csquotes}: \enquote{doppelte Anführungszeichen},
\enquote*{einfache Anführungszeichen}, \enquote{ein \enquote{Zitat} im Zitat}. Für Zitate mit Literaturverweis gibt es
\verb|\textquote| bzw. \verb|\textcquote|: \textquote[{\cite[202]{spiegelberg}}]{Ein Zitat mit Verweis}, noch einfacher
\textcquote[202]{spiegelberg}{Ein Zitat mit Verweis}.

Für längere Zitate verwenden Sie \emph{displayquote} oder \emph{displaycquote}:

\begin{displayquote}[{\cite[202]{spiegelberg}}]
	Ein langes langes langes langes langes langes langes langes langes langes langes langes langes langes langes langes langes langes
	langes langes langes langes langes langes langes langes langes langes langes langes langes langes langes langes langes langes langes langes
	langes langes langes langes langes langes langes langes langes langes langes langes langes langes langes langes langes langes langes langes
	Zitat.
\end{displayquote}

\begin{displaycquote}[202]{spiegelberg}
	Ein langes langes langes langes langes langes langes langes langes langes langes langes langes langes langes langes langes langes
	langes langes langes langes langes langes langes langes langes langes langes langes langes langes langes langes langes langes langes langes
	langes langes langes langes langes langes langes langes langes langes langes langes langes langes langes langes langes langes langes langes
	Zitat.
\end{displaycquote}
%
Das Paket \emph{csquotes} stellt auch Makros für Auslassungen in Zitaten zur Verfügung: \textelp{}, für Einfügungen: \textins{meine Einfügung}
bzw. für beides kombiniert: \textelp{Einfügung nach Auslassung}, \textelp*{Einfügung vor Auslassung}.

Verwenden Sie statt manueller Textauszeichnung linguistischer Ebenen möglichst das semantische Markup, das die Klasse zur Verfügung stellt:
\begin{itemize}
	\item Ausdruck (Objektsprachliches) kursiv: Das Wort \Expression{Wort}.
	\item Bedeutungsangaben in einfachen Anführungszeichen: \Meaning{Bedeutung}
	\item Semantische Konzepte in Kapitälchen: Das Konzept \Concept{Konzept}
\end{itemize}
%
Nummerierte linguistische Beispiele bekommen Sie mit der Umgebung \emph{example} (für mehrzeilige Beispiele) und \emph{examples} (für einzeilige):

\begin{example}
	Das ist ein mehrzeiliges Beispiel
	
	Es kann mehrere Absätze enthalten
\end{example}

\begin{examples}
	\item Das ist ein einzeiliges Beispiel\label{exa:Beispiel-einzeilig}
	\item Jeder Absatz wird in diesem Stil neu nummeriert
\end{examples}
%
Auf die Beispiele verweist man am Besten so: \prettyref{exa:Beispiel-einzeilig}. Übrigens auch auf Kapitel: \prettyref{cha:hinweise}.

% Die Bibliographie
\printbibliography[heading=bibnumbered]

\end{document}
