% Example file for the class ``univie-ling-paper''
% Run pdflatex -- biber -- pdflatex
\documentclass[naustrian,english]{univie-ling-paper}

\usepackage{babel}

% By default, the class loads the biblatex package
% with Unified style and biber backend.
% If you want to use classic bibtex (with unified.bst),
% use the class option ``biblatex=false''
\bibliography{biblatex-examples}

% Personal data
\author{First Name Surname, BA}
\matrikelnr{0000000}
\studienkennzahl{A 066 899}

% Joint work:
%\author{Vorname Nachname, BA \and Vorname Nachname, BA}
%\matrikelnr{0000000 \and 1111111}
%\studienkennzahl{A 066 899 \and A 066 899}

% Paper-related data
\title{Title of the Paper}
\course{ps}{Course Title}% valid options: ps, se, vo, pv, ue, ko
\semester{ss}{2015}% valid options: ss, ws
\instructor{Univ.-Prof. Dr. First Name Surname}

% The paper type is automatically set for ps and se
% Comment out and adjust if needed
%\texttype{Forschungsbericht}

\begin{document}

% The title page
\maketitle

% Anti-plagiarism declaration
\makedeclaration

\tableofcontents

\section{General hints}\label{sec:hints}

It is recommended that you use \textsf{biblatex} for consistent references in Unified style
(see \foreignlanguage{naustrian}{\emph{Leitfaden für die Gestaltung von schriftlichen Arbeiten und Unterlagen}; März 2018}).
For normal inline references, use \verb|\textcite|: \textcite[22]{sarfraz}; for references in parentheses use \verb|\parencite|:
\parencite{sarfraz}.

For quotation marks and quotations, use the macros provided by the package \emph{csquotes}: \enquote{double quotation marks},
\enquote*{single quotation marks}, \enquote{an \enquote{embedded} quotation}. Quotation with references can be produced with
\verb|\textquote| and \verb|\textcquote|: \textquote[{\cite[222]{sarfraz}}]{A quote with reference}, even simpler
\textcquote[222]{sarfraz}{A quote with reference}.

For longer quotations, use \emph{displayquote} or \emph{displaycquote}:

\begin{displayquote}[{\cite[246]{sarfraz}}]
	A long long long long long long long long long long long long long long long long long long
	long long long long long long long long long long long long long long long long long long long long
	long long long long long long long long long long long long long long long long long long long long
	quotation.
\end{displayquote}

\begin{displaycquote}[246]{sarfraz}
	A long long long long long long long long long long long long long long long long long long
	long long long long long long long long long long long long long long long long long long long long
	long long long long long long long long long long long long long long long long long long long long
	quotation.
\end{displaycquote}
%
There are also macros for omissions: \textelp{} and insertions in quotations: \textins{my insertion} as well as for both combined:
\textelp{insertion after omission}, \textelp*{insertion before omission}.

Use semantic markup instead of manual shape change:
\begin{itemize}
	\item Expressions (object language) in italics: The word \Expression{Wort}.
	\item Meaning is given in single quotes: \Meaning{meaning}
	\item Semantic concepts in small caps: the concept \Concept{concept}
\end{itemize}
%
Numbered linguistic examples can be produced by means of the \emph{example} environment (for multi-line examples) and \emph{examples}
environment (for subsequent single-line examples):

\begin{example}
	This is a multi-line example
	
	It might consist of several paragraphs
\end{example}

\begin{examples}
	\item This is a single-line example\label{exa:example-single}
	\item Each paragraph gets its own number
\end{examples}
%
It is recommended to refer to the examples like this: \prettyref{exa:example-single}. Also to sections\footnote{Nota bene: papers do not
	have chapters, just sections. Only books (monographs) have chapters!}, incidentally: \prettyref{sec:hints}.

% The bibliography
\clearpage
\printbibliography[heading=bibnumbered]

\end{document}
