% Beispieldatei für die Klasse ``univie-ling-paper''
% Übersetzen mit pdflatex -- biber -- pdflatex
\documentclass[naustrian]{univie-ling-paper}

\usepackage{babel}

% Standardmäßig wird das Biblatex-Paket mit Unified-Stil und
% Biber-Backend geladen.
% Wenn Sie klassisches Bibtex (mit unified.bst) verwenden wollen,
% verwenden Sie die Klassenoption ``biblatex=false''
\bibliography{biblatex-examples}

% Persönliche Angaben
\author{Vorname Nachname, B.A.}
\matrikelnr{0000000}
\studienkennzahl{A 066 899}

% Mehrere Autoren:
%\author{Vorname Nachname, B.A. \and Vorname Nachname, B.A. }
%\matrikelnr{0000000 \and 1111111}
%\studienkennzahl{A 066 899 \and A 066 899}

% Arbeitsrelevante Angaben
\title{Titel der Arbeit}
\course{ps}{Seminartitel}% Mögliche Optionen: ps, se, vo, pv, ue, ko
\semester{ss}{2015}% Mögliche Optionen: ss, ws
\instructor{Univ.-Prof. Dr. Vorname Name}

% Der Arbeitstyp wird für ps und se automatisch gesetzt.
% Kommentieren Sie, falls nötig, das folgende aus, um einen
% Arbeitstypen zu definieren
%\texttype{Forschungsbericht}

\begin{document}

% Die Titelseite
\maketitle

% Selbstaendigkeitserklaerung
\makedeclaration

\tableofcontents

\section{Ein paar Tipps}\label{sec:hinweise}

Verwenden Sie am Besten \textsf{biblatex} für konsistente Literaturverweise im \emph{Unified}-Stil (vgl. \emph{Leitfaden für die Gestaltung von
schriftlichen Arbeiten und Unterlagen}; März 2018).
Für normale Verweise im Text \verb|\textcite|: \textcite[22]{brandt}, für Verweise in Klammern \verb|\parencite|: \parencite{brandt}.

Für Anführungszeichen und Zitate verwenden Sie am Besten die Befehle des Paketes \emph{csquotes}: \enquote{doppelte Anführungszeichen},
\enquote*{einfache Anführungszeichen}, \enquote{ein \enquote{Zitat} im Zitat}. Für Zitate mit Literaturverweis gibt es
\verb|\textquote| bzw. \verb|\textcquote|: \textquote[{\cite[202]{spiegelberg}}]{Ein Zitat mit Verweis}, noch einfacher
\textcquote[202]{spiegelberg}{Ein Zitat mit Verweis}.

Für längere Zitate verwenden Sie \emph{displayquote} oder \emph{displaycquote}:

\begin{displayquote}[{\cite[202]{spiegelberg}}]
	Ein langes langes langes langes langes langes langes langes langes langes langes langes langes langes langes langes langes langes
	langes langes langes langes langes langes langes langes langes langes langes langes langes langes langes langes langes langes langes langes
	langes langes langes langes langes langes langes langes langes langes langes langes langes langes langes langes langes langes langes langes
	Zitat.
\end{displayquote}

\begin{displaycquote}[202]{spiegelberg}
	Ein langes langes langes langes langes langes langes langes langes langes langes langes langes langes langes langes langes langes
	langes langes langes langes langes langes langes langes langes langes langes langes langes langes langes langes langes langes langes langes
	langes langes langes langes langes langes langes langes langes langes langes langes langes langes langes langes langes langes langes langes
	Zitat.
\end{displaycquote}
%
Es gibt auch Makros für Auslassungen: \textelp{} und für Einfügungen in Zitaten: \textins{meine Einfügung} bzw. für beides kombiniert:
\textelp{Einfügung nach Auslassung}, \textelp*{Einfügung vor Auslassung}.

Verwenden Sie semantisches Markup statt manueller Textauszeichnung:
\begin{itemize}
	\item Ausdruck (Objektsprachliches) kursiv: Das Wort \Expression{Wort}.
	\item Bedeutungsangaben in einfachen Anführungszeichen: \Meaning{Bedeutung}
	\item Semantische Konzepte in Kapitälchen: Das Konzept \Concept{Konzept}
\end{itemize}
%
Nummerierte linguistische Beispiele bekommen Sie mit der Umgebung \emph{example} (für mehrzeilige Beispiele) und \emph{examples} (für einzeilige):

\begin{example}
	Das ist ein mehrzeiliges Beispiel
	
	Es kann mehrere Absätze enthalten
\end{example}

\begin{examples}
	\item Das ist ein einzeiliges Beispiel\label{exa:Beispiel-einzeilig}
	\item Jeder Absatz wird in diesem Stil neu nummeriert
\end{examples}
%
Auf die Beispiele verweist man am Besten so: \prettyref{exa:Beispiel-einzeilig}. Übrigens auch auf Abschnitte\footnote{Denken Sie dran: Seminararbeiten haben,
	wie Artikel, keine \emph{Kapitel}, sondern nur \emph{Abschnitte}. Kapitel haben Bücher!}: \prettyref{sec:hinweise}.

% Die Bibliographie
\clearpage
\printbibliography[heading=bibnumbered]

\end{document}
