% !TeX spellcheck = en_US
\documentclass[english]{article}

\usepackage[osf]{libertine}
\usepackage[scaled=0.7]{beramono}
\usepackage[T1]{fontenc}
\usepackage[latin9]{inputenc}
\usepackage[hyphens]{url}
\usepackage{csquotes}
\usepackage[pdfusetitle,
 bookmarks=true,bookmarksnumbered=false,bookmarksopen=false,
 breaklinks=false,pdfborder={0 0 0},backref=false,colorlinks=false]
 {hyperref}

% Tweak the TOC (make it more compact)
\usepackage{tocloft}
\setlength{\cftaftertoctitleskip}{6pt}
\setlength{\cftbeforesecskip}{0pt}
\setlength{\cftbeforesubsecskip}{0pt}
\renewcommand{\cfttoctitlefont}{\normalsize\bfseries}
\renewcommand{\cftsecfont}{\small\bfseries}
\renewcommand{\cftsecpagefont}{\small\bfseries}
\renewcommand{\cftsubsecfont}{\small}
\renewcommand{\cftsubsecpagefont}{\small}

% markup
\newcommand*\jmacro[1]{\textbf{\texttt{#1}}}
\newcommand*\jcsmacro[1]{\jmacro{\textbackslash{#1}}}
\newcommand*\joption[1]{\textbf{\texttt{#1}}}
\newcommand*\jfmacro[1]{\texttt{#1}}
\newcommand*\jfcsmacro[1]{\jfmacro{\textbackslash{#1}}}

% macros
\newcommand*\uvlt{\textsf{univie-ling biblatex}}
\providecommand{\LyX}{\texorpdfstring{L\kern-.1667em\lower.25em\hbox{Y}\kern-.125emX\@}{LyX}}

% improve layout
\tolerance 1414
\hbadness 1414
\emergencystretch 1.5em
\hfuzz 0.3pt
\widowpenalty = 10000
\vfuzz \hfuzz
\raggedbottom

% Conditional pagebreak
\def\condbreak#1{%
	\vskip 0pt plus #1\pagebreak[3]\vskip 0pt plus -#1\relax}

\usepackage{microtype}

\usepackage{babel}

\usepackage{listings}
\lstset{language={[LaTeX]TeX},
        basicstyle={\small\ttfamily},
        frame=single}

\setcounter{tocdepth}{2}

\begin{document}

\title{The \uvlt\ styles}

\author{\texorpdfstring{J�rgen Spitzm�ller%
\thanks{Please report issues via \protect\url{https://github.com/jspitz/univie-ling}.}}{J�rgen Spitzm�ller}}

\date{Version 2.9, 2024/09/28}

\maketitle

\begin{abstract}
\noindent The \uvlt\ styles provide citation and bibliography styles that conform to the style guide of the Linguistic Department
of Vienna University which itself follows the \emph{Unified Style Sheet for Linguistic Papers}.
This manual documents the style.
\end{abstract}

\tableofcontents

\section{Aims and scope}

The \uvlt\ styles include a \textsf{biblatex} bibliography style (\texttt{univie-ling.bbx}) and a citation style
(\texttt{univie-ling.cbx}) that implement the citation conventions common in Viennese (Applied) Linguistics.
The styles draw on the \emph{Unified Style Sheet for Linguistic Papers}.%
\footnote{See \url{https://www.linguisticsociety.org/sites/default/files/style-sheet_0.pdf}.}

\section{Requirements of \uvlt}\label{sec:req-jslp}

The following class and packages are required and loaded by \uvlt:
\begin{itemize}
 \setlength\itemsep{0pt}
 \item \textsf{biblatex}: Contemporary bibliography support.
 \item \textsf{biber}: Bibliography processor for biblatex.
\end{itemize}

\section{Use}\label{use}

There is not much to say here. The styles are used as any other \textsf{biblatex} styles: You load them via the
\joption{style} option of biblatex (\joption{style=univie-ling} in this case), pass additional options here as well
and use the citation and bibliography commands documented in the \textsf{biblatex} manual \cite{bibltx}.

\section{Style-specific Options}\label{options}

On top of the options provided by the \textsf{biblatex} package itself (see \cite{bibltx}), the \uvlt\ styles provide
a couple of specific options:
\begin{description}
 \setlength\itemsep{0pt}
 \item{\joption{annotations=true|false}}: if \joption{true}, the contents of the \texttt{annotation} or \texttt{annot} field
      of your Bib\TeX\ entry are output (in a separate indented paragraph that follows the entry). Use this to generate annotated
      bibliographies. The options is \joption{false} by default.
      
      Annotations might also be stored in separate files as documented in \cite[sec.~3.14.8]{bibltx}.
  \item{\joption{issueeditor=true|false}}: if \joption{true}, editors of journal special issues are output.
      The options is \joption{false} by default.
\end{description}

\section{Release History}

\begin{description}
  \item 2024/09/28 (v.\,2.9)
    \begin{itemize}
	   \item Add option \joption{annotations} which, if \joption{true},
	         will print out the \texttt{annotation} field
	         (useful for annotated bibliographies).
	   \item Add this manual.
	 \end{itemize}
  \item 2024/09/20 (v.\,2.8)
  \item 2024/07/23 (v.\,2.7)
  \item 2024/06/27 (v.\,2.6)
  \item 2024/05/09 (v.\,2.5)
  \item 2023/03/31 (v.\,2.4)
  \item 2023/01/26 (v.\,2.3)
  \item 2022/12/06 (v.\,2.2)
  \item 2022/10/21 (v.\,2.1)
  \item 2022/10/02 (v.\,2.0)
  \item 2022/09/08 (v.\,1.20)
     \begin{itemize}
	     \item No change to these styles.
     \end{itemize} 
  \item 2022/06/18 (v.\,1.19)
    \begin{itemize}
 	     \item properly re-define a macro in the citations style.
 	\end{itemize}
  \item 2022/05/11 (v.\,1.18)
     \begin{itemize}
 	     \item No change to these styles.
     \end{itemize}
  \item 2022/02/05 (v.\,1.17)
    \begin{itemize}
 	     \item Draw on authoryear-comp rather than uncompressed
               version (in line with major journal's interpretation
               of the unified style sheet) 
    \end{itemize}
  \item 2021/11/03 (v.\,1.16)
    \begin{itemize}
 	   \item No change to these styles.
    \end{itemize}
  \item 2021/10/19 (v.\,1.15)
    \begin{itemize}
 	   \item Output url date only if url is given (not with DOI).
        \item Fix emphasizing of journal title.
    \end{itemize}
  \item 2021/09/01 (v.\,1.14)
    \begin{itemize}
 	   \item Sentence-case journal issue title and online title.
       \item Emphasize journal subtitle.
    \end{itemize}
  \item 2020/11/11 (v.\,1.13)
    \begin{itemize}
   	   \item Fix whitespace issue in thematic issue string.
   	\end{itemize}
  \item 2020/06/25 (v.\,1.12)
    \begin{itemize}
 	   \item only output origyear if it differs from year.
 	\end{itemize}
  \item 2020/05/05 (v.\,1.11)
     \begin{itemize}
 	   \item No change to these styles.
     \end{itemize}
  \item 2019/05/01 (v.\,1.10)
      \begin{itemize}
      	\item only print origyear (not full origdate) in label.
 	    \item fix title capitalization in inbook, mvbook and mvcollection types.
 	  \end{itemize}
  \item 2019/01/21 (v.\,1.9)
     \begin{itemize}
     	\item correct fix of duplicate editorstrg parentheses.
     \end{itemize}
  \item 2019/01/15 (v.\,1.8)
    \begin{itemize}
 	  \item fix another whitespace issue.
      \item fix duplicate editorstrg parentheses.
    \end{itemize}
  \item 2018/11/07 (v.\,1.7)
    \begin{itemize}
    	\item fix some whitespace issues.
     \end{itemize} 
  \item 2018/11/04 (v.\,1.6)
      \begin{itemize}
      	\item univie-ling.bbx: fix bookauthor with inbook type.
     \end{itemize}
  \item 2018/09/03 (v.\,1.5)
    \begin{itemize}
 	   \item No change to these styles.
    \end{itemize}
  \item 2018/03/02 (v.\,1.4)
    \begin{itemize}
  	    \item Fix journal dates (if month and day is given).
    \end{itemize}
  \item 2018/03/02 (v.\,1.3)
    \begin{itemize}
 	  \item Fix whitespace issue in \texttt{@inproceedings} entry.
       \item Specify date output details (era, uncertain etc.).
    \end{itemize}
  \item 2018/02/13 (v.\,1.2)
    \begin{itemize}
  	    \item No change to these styles.
  	\end{itemize}
  \item 2018/02/11 (v.\,1.1)
	\begin{itemize}
		\item Fixed location list output wrt the Unfied Style Sheet.
	    \item Link DOIs to preferred resolver (thanks, Katrin Leinweber!)
	\end{itemize}
  \item 2018/02/08 (v.\,1.0)
	\begin{itemize}
		\item Initial release on CTAN.
	\end{itemize}
\end{description}

\begin{thebibliography}{1}

\bibitem{bibltx} Lehman, Philipp (with Audrey Boruvka, Philip Kime and Joseph Wright):
\emph{The biblatex Package. Programmable Bibliographies	and Citations}. March 3, 2016.
\url{http://www.ctan.org/pkg/biblatex}.

\end{thebibliography} 

\end{document}
