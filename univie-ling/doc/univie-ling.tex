\documentclass[english]{article}

\usepackage[osf]{libertine}
\usepackage[scaled=0.7]{beramono}
\usepackage[T1]{fontenc}
\usepackage[latin9]{inputenc}
\usepackage[hyphens]{url}
\usepackage[pdfusetitle,
 bookmarks=true,bookmarksnumbered=false,bookmarksopen=false,
 breaklinks=false,pdfborder={0 0 0},backref=false,colorlinks=false]
 {hyperref}

\usepackage[dvipsnames]{xcolor}
\usepackage{MnSymbol}
% macros
\newcommand*\uvlt{\textsf{univie-ling}}
\newcommand*\uvlc[1]{\textsf{univie-ling-#1}}
\newcommand*\mlink[1]{\textcolor{Maroon}{\hyperref{#1}{}{}{\textbf{$\blacktriangleright$ Manual}}}}

% improve layout
\tolerance 1414
\hbadness 1414
\emergencystretch 1.5em
\hfuzz 0.3pt
\widowpenalty = 10000
\vfuzz \hfuzz
\raggedbottom

% Conditiona pagebreak
\def\condbreak#1{%
	\vskip 0pt plus #1\pagebreak[3]\vskip 0pt plus -#1\relax}

\usepackage{microtype}

\usepackage{babel}


\begin{document}

\title{The \uvlt\ bundle}

\author{\texorpdfstring{J�rgen Spitzm�ller%
\thanks{Please report issues via \protect\url{https://github.com/jspitz/univie-ling}.}}{J�rgen Spitzm�ller}}

\date{Version 1.20, 2022/09/08}

\maketitle

\section{Aim}

The \uvlt\ bundle consists of a set of classes, and a \textsf{biblatex} bibliography and citation style, useful for students and academics
of linguistics at the University of Vienna, Austria (\url{https://linguistics.univie.ac.at}). The classes follow the corporate design
of the university and institutional regulations (as far as these exist).

This manual briefly introduces these classes and styles. Please refer to the linked manuals of the respective class for details.

\section{Classes and Styles}

\subsection{\uvlc{expose}}

The \uvlc{expose} class provides a \LaTeXe\ class suitable for those research proposals (\emph{Expos�s}) that are required in the context of the public presentation of a dissertation project (\emph{F�P}) at the University of Vienna. \mlink{univie-ling-expose.pdf}

\subsection{\uvlc{handout}}

The \uvlc{handout} class provides a \LaTeXe\ class suitable for handouts that accompany presentations in classes or at conferences.
The class adheres to the corporate design of the University of Vienna (although no dedicated specs for handouts
are provided there).
Therefore, although this class has been written for students in the Department of Linguistics, it might also be useful for other fields
and for students and researchers alike. \mlink{univie-ling-handout.pdf}

\subsection{\uvlc{paper}}

The \uvlc{paper} class provides a \LaTeXe\ class suitable for papers (i.\,e., [\emph{Pro}]\emph{Seminararbeiten}) in (Applied)
Linguistics at the Department of Linguistics, University of Vienna. \mlink{univie-ling-paper.pdf}

\subsection{\uvlc{thesis}}

The \uvlc{thesis} class provides a \LaTeXe\ class suitable for Bachelor's, Master's, Diploma and Doctoral theses in
(Applied) Linguistics at the Department of Linguistics, University of Vienna. \mlink{univie-ling-thesis.pdf}

\subsection{\uvlc{wlg}}

The \uvlc{wlg} class provides a \LaTeXe\ class suitable for articles in the journal \emph{Wiener Linguistische Gazette} (WLG),
the house journal of the Department of Linguistics at the University of Vienna (\url{https://wlg.univie.ac.at}).
\mlink{univie-ling-wlg.pdf}

\subsection{\uvlt\ Bibliography and Citation Styles}

The bundle also includes a \textsf{biblatex} bibliography style (\texttt{univie-ling.bbx}) and a citation style
(\texttt{univie-ling.cbx}) that implement the citation conventions common in Viennese (Applied) Linguistics.
The style draws on the \emph{Unified Style Sheet for Linguistic Papers}.%
\footnote{See \url{https://www.linguisticsociety.org/sites/default/files/style-sheet_0.pdf}.}
Documentation is included in the documentation of the classes above.

\end{document}
